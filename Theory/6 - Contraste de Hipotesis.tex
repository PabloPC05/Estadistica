\section{Contraste de Hipótesis}

\subsection{Principios básicos de un contraste de hipótesis}

\ejemplo{
    Supongamos que en un laboratorio se está estudiando cierta reacción química sobre una determinada sustancia y que el resultado de dicha reacción es una variable observable que se puede modelizar mediante una v.a. $X$ con distribución normal. \\ \\ Por experiencias anteriores se sabe que, si en la sustancia está presente cierto mineral, $X \sim N(\mu=10, \sigma=4)$ y si no lo está $X \sim N(\mu=11, \sigma=4)$. Se puede comprobar por medio de unos análisis si el mineral está o no presente en la sustancia en estudio, pero dichos análisis son muy costosos, por lo que se procede a realizar la reacción química $n=25$ veces para decidir, a la luz de los resultados, si $\mu=10$ o $\mu=11$
}

\begin{definición} [Hiptótesis Estadística]
Una hipótesis estadística es cualquier afirmación acerca de un modelo estadístico.
\end{definición}


\begin{definición} [Hipótesis Estadística Simple y Compuesta]
    Una hipótesis estadística es simple si especifica totalmente el modelo estadístico, en otro caso, se dice que es compuesta
\end{definición}

\ejemplo{
    Sea $X \sim N(\mu, \sigma^{2}) \implies$
    \begin{enumerate}
        \item $H_0: \mu = 10 \text{ y } \sigma^2 = 4 \rightarrow$ simple 
        \item $H_0: \mu \in [9,11] \text{ y } \sigma^2 = 4 \rightarrow$ compuesta
        \item $H_0: \mu = ? \text{ y } \sigma^2 = 4 \rightarrow$ compuesta
        \item $H_0: \mu \leq 9 \text{ y } \sigma^2 = 4 \rightarrow$ compuesta
    \end{enumerate}
}

\begin{definición} [Hipótesis Estadística Nula y Alternativa]
    Se dice hipótesis nula a la afirmación inicial o por defecto que se pone a prueba. Se asume cierta hasta que haya suficiente evidencia para rechazarla. \\
    En contraste la hipótesis alternativa es la afirmación que se quiere demostrar o detectar. 
\end{definición}

\ejemplo{
    Imagina una fábrica que proue botellas de agua de $1000ml$ de capacidad. Para asegurar la calidad, se toma una muestra aleatoria de las botellas y se mide su contenido. Por tanto nuestro objetivo es verificar si la máquina está llenando correctamente las botellas o si hay un problema. 
    \begin{itemize}
        \item Hipótesis nula: $H_0: \mu = 1000ml$ (la máquina está funcionando correctamente)
        \item Hipótesis alternativa: $H_1: \mu \neq 1000ml$ (la máquina no está funcionando correctamente)
    \end{itemize}
    En este caso concreto sería una prueba \textbf{bilateral} porque nos preocupa tanto si las botellas están con menos como con más de 1 litro. 
}

\begin{definición} [Contraste de Hipótesis Paramétrico]
    Un contraste estadístico es cualquier particiión del espacio muestral $\chi^{n}$ en dos subconjutnos $RA$ y $RC$ de tal manera que si el punto muestral $\vec{x} = (x_1, \ldots, x_n)$ pertenece a $RA$ se dice que se acepta la hipótesis nula, es decir, se admite $H_0: \theta \in \Theta_0$ y si $\vec{x} \in RC$ se dice que se rechaza la hipótesis nula o equivalentemente que se acepta la hipótesis alternativa, es decir, se admite $H_1: \theta \in \Theta_1$. \\ 
    A $RA$ se le denomina \textbf{región de aceptación} y a $RC$ se le denomina \textbf{región crítica}. 
\end{definición}

\ejemplo{
    En el primer ejemplo anterior $\Theta=\left\{\mu_{0}, \mu_{1}\right\}, \Theta_{0}=\left\{\mu_{0}\right\}, \Theta_{1}=\left\{\mu_{1}\right\}$
}

\begin{definición}[Estadístico de Contraste]
    En un problema de contraste de hipótesis, se pretende contrastar $H_0$ frente a $H_1$. La decisión ha de basarse en la evidencia aportada por la observación de una muestra o equivalentemente por la observación de un cierto estadístico $T$ denominado \underline{estadístico del contraste} que será usualmente un estimador suficiente del parámetro $\theta$.
\end{definición}

\ejemplo{
    Siguiendo con el ejemplo anterior de la fábrica de botellas, se puede ver que la máquina embotelladora está malfuncionando de dos formas: 
    \begin{itemize}
        \item Tomando una muestra: Supongamos que se toman 3 botellas:
        $$X = (999, 1002, 1005)$$
        Se define una región crítica sobre la muestra completa, por ejemplo: decidimos rechazar $H_0$ si al enos dos botelllas tienen mas de 1003ml  si la mínima es mayor que 500ml. \\
        En este caso en concreto, sólo una tiene mas de 503ml y la minima es menor de 500ml, por tanto no se rechaza la hipótesis nula o $H_0$.
        \item Tomando un estadístico: Tomemos el estadístico media muestral y el ejemplo anterior. En este caso la media muestral es:
        $$\bar{x} = \frac{999 + 1002 + 1005}{3} = 1002$$
        Supongamos que sabemos que en este caso la region crítica es que $\bar{x} > 1003$. En este caso, como $\bar{x} < 1003$, no se rechaza la hipótesis nula o $H_0$.
    \end{itemize}
}

\begin{observación}
    El contraste de hipótesis basado en un estadístico, exige conocer la distribución de dicho estadístico para los posibles valores del parámetro. El contraste se basa en ver si el valor observado del estadístico es raro bajo esa distribución, si ocurre un valor "muy raro" existen dos posibilidades: fue pura casualidad (poco probable) o más probablemente $H_0$ es falsa. 
\end{observación}

\begin{definición} [Región Crítica]
    Sea una partición del espacio muestral $\chi^{n}$ en dos subconjuntos $C$ y $C^{*}$ tales que $\chi^{n}=C \bigcup C^{*}$ y $C \bigcap C^{*}=\phi$. $C$ es una región crítica para el contraste $H_{0}: \theta \in \Theta_{0}$ frente a $H_{1}: \theta \in \Theta_{1}$ sí y sólo sí, se rechaza $H_{0}$ cuando se observa un valor muestral $\left(x_{1}, \cdots, x_{n}\right) \in C$, en cuyo caso se acepta $H_{1}$. Consecuentemente, $C^{*}$ se denomina región de aceptación y si $\left(x_{1}, \cdots, x_{n}\right) \in C^{*}$, se dice que no hay suficiente evidencia estadística para rechazar $H_{0}$, en este sentido se acepta $H_{0}$
\end{definición}

\subsection{Errores de tipo I y de tipo II}
\begin{definición}[Error de tipo I]
    El error de tipo I es el error que se comete cuando se rechaza $H_{0}$ siendo cierta. La probabilidad de cometer este error se denomina \textbf{nivel de significación del test} y se denota por $\alpha=P_{\theta_{0}}(C)$, donde $\theta_{0} \in \Theta_{0}$
\end{definición}

\begin{definición}[Error de tipo II]
    El error de tipo II es el error que se comete cuando se acepta $H_{1}$ siendo falsa. La probabilidad de cometer este error se denomina \textbf{potencia del test} y se denota por $\beta=P_{\theta_{1}}(C)$, donde $\theta_{1} \in \Theta_{1}$
\end{definición}

\ejemplo{
    En el primer ejemplo de todos, las probabilidades de cometer error de tipo I y error de tipo II son $P(\mathrm{I})=P(\bar{x} \geq k \mid \mu=10)$ y $P(\mathrm{II})=P(\bar{x}<k \mid \mu=11)$
}

\begin{observación}
    Lo idóneo sería contar con un test de baja probabilidad de cometer errores tanto de tipo I como de tipo II, pero en la práctica si bajas un error el otro suele subir de forma equilibrida. \\
    La única forma realista de reducir ambos errores simultáneamente sería tomar una muestra más grande, pero ésto conllevaría mas costes, tiempo y recursos. 
\end{observación}

\begin{definición} [Función de Potencia]
Si $C$ es una región crítica para el contraste $H_{0}: \theta \in \Theta_{0}$ frente a $H_{1}: \theta \in \Theta_{1}$, se define la función de potencia del test como la función $\beta_{C}: \Theta \rightarrow[0,1]$ que a cada valor $\theta$ del parámetro le asigna el valor $\beta_{C}(\theta)=P_{\theta}(C)$, es decir, la probabilidad de rechazar $H_{0}$ cuando el valor del parámetro es $\theta$
\end{definición}

\begin{definición} [Nivel de significación y tamaño del test ]
Un test $C$ tiene nivel de significación $\alpha \in[0,1]$ sí y sólo sí $\sup \beta_{C}(\theta) \leq \alpha$ y se denomina tamaño del test al valor $\sup \beta_{C}(\theta)$ $\theta \in \Theta_{0}$
\end{definición}

\ejemplo{
En el ejemplo anterior, con $n=25, C=\{10<\bar{x}<10,006\}$, $P(I)=\beta_{C}(\mu=10)=P(10<\bar{x}<10.006 \mid \mu=10)=0.05$ $P(\mathrm{II})=1-\beta_{C}(\mu=11)=P(10<\bar{x}<10.006 \mid \mu=11)=0.976$

En el ejemplo anterior, con $n=25, C=\{\bar{x} \geq k\}$ y $\alpha=0.05$, $P(\mathrm{I})=\beta_{C}(\mu=10)=P(\bar{x} \geq 11.316 \mid \mu=10)=0.05$ $P(\mathrm{II})=1-\beta_{C}(\mu=11)=P(\bar{x}<11.316 \mid \mu=11)=0.6554$

En el ejemplo anterior, con $n=100, C=\{\bar{x} \geq k\}$ y $\alpha=0.05$,\\
$P(\mathrm{I})=\beta_{C}(\mu=10)=P(\bar{x} \geq 10.658 \mid \mu=10)=0.05$\\
$P(\mathrm{II})=1-\beta_{C}(\mu=11)=P(\bar{x}<10.658 \mid \mu=11)=0.196$
    
}

\begin{definición} [p-valor]
 Si para contrastar $H_{0}: \theta \in \Theta_{0}$ frente a $H_{1}: \theta \in \Theta_{1}$, el test tiene región crítica 
 $$C=\left\{\left(x_{1}, \ldots, x_{n}\right): T\left(x_{1}, \ldots, x_{n}\right) \geq k\right\}$$
 para $T$ un estadístico conveniente, y se observa la muestra $\left(x_{1}, \ldots, x_{n}\right)$, se denomina p -valor correspondiente a $\left(x_{1}, \ldots, x_{n}\right)$ al valor

$$
p\left(x_{1}, \ldots, x_{n}\right)=\sup _{\theta \in \Theta_{0}} P\left\{T\left(X_{1}, \ldots, X_{n}\right) \geq T\left(x_{1}, \ldots, x_{n}\right) \mid \theta\right\}
$$

Si el tamaño del test es $\alpha$ y observada la muestra $\left(x_{1}, \ldots, x_{n}\right)$ el p-valor correspondiente $p\left(x_{1}, \ldots, x_{n}\right) \leq \alpha$, entonces $\left(x_{1}, \ldots, x_{n}\right)$ pertenece a la región crítica y por lo tanto se rechaza $H_{0}$. Si el p -valor $p\left(x_{1}, \ldots, x_{n}\right)>\alpha$, entonces $\left(x_{1}, \ldots, x_{n}\right)$ pertenece a la región de aceptación y por lo tanto no hay suficiente evidencia estadística para rechazar $H_{0}$
\end{definición}

\ejemplo{
En el ejemplo anterior, con $n=100, C=\{\bar{x} \geq k\}$ y $\alpha=0.05$,\\
$P(I)=\beta_{c}(\mu=10)=P(\bar{x} \geq 10.658 \mid \mu=10)=0.05$\\
$P(\mathrm{II})=1-\beta_{C}(\mu=11)=P(\bar{x}<10.658 \mid \mu=11)=0.196$\\
Entonces, observada $\bar{x}=11$

$$
p(11)=P(\bar{X} \geq 11 \mid \mu=10)=P(Z \geq 2.5)=0.00621
$$

Por lo tanto, se rechaza $H_{0}: \mu=10$ a favor de $H_{1}: \mu=11$    
}


\begin{observación}
\vspace{-2.5em}
\begin{enumerate}
\item n y $\alpha$ son valores fijados de antemano
\item Las hipótesis nula y alternativa no son intercambiables puesto que el tratamiento que reciben es asimétrico, la asimetría queda matizada por el valor $\alpha$ que se elija como nivel de significación y por la probabilidad de error de tipo II que resulte una vez diseñado el test, pues podría ocurrir que para $\theta \in \Theta_{1}, P_{\theta}\left(C^{c}\right)=1-\beta_{C}(\theta) \leq \alpha$
\item  En el contraste de hipótesis planteado se considera $H_{0}$ como la hipótesis de interés, en el sentido que para poder invalidarla es necesario esgrimir una gran evidencia. Por consiguiente, los test de hipótesis se emplean con un carácter conservador, a favor de la hipótesis nula, ya que el nivel de significación que se fija, intenta garantizar que sea muy infrecuente rechazar una hipótesis nula correcta, y la preocupación por dejar vigente una hipótesis nula falsa es menor, pudiéndose aceptar en este último caso riesgos más altos. En este sentido, si el resultado de un contraste de hipótesis es aceptar $H_{0}$, debe interpretarse que las observaciones no han aportado suficiente evidencia para descartarla; mientras que, si se rechaza, es porque se está razonablemente seguro de que $H_{0}$ es falsa y, por consiguiente, aceptamos $H_{1}$
\end{enumerate}
\end{observación}

\begin{proposición} [Criterio de comparación de contrastes]
Si $C$ y $C^{\prime}$ son dos test con nivel de significación $\alpha$ basados en una muestra $\left(X_{1}, \cdots X_{n}\right)$ de $\left\{F_{\theta}, \theta \in \Theta\right\}$, para contrastar $H_{0}: \theta \in \Theta_{0}$ frente a $H_{1}: \theta \in \Theta_{1}$, tales que $\beta_{C}(\theta) \geq \beta_{C^{\prime}}(\theta), \forall \theta \in \Theta_{1}$, entonces $C$ es uniformemente más potente que $C^{\prime}$    
\end{proposición}



\subsection{Test uniformemente más potente de tamaño $\alpha$}


\begin{proposición}
Sea $C$ una región crítica para el contraste $H_{0}: \theta \in \Theta_{0}$ frente a $H_{1}: \theta \in \Theta_{1}$, basada en una muestra ( $X_{1}, \cdots X_{n}$ ) de $\left\{F_{\theta}, \theta \in \Theta\right\}$ $C$ es un test uniformemente de máxima potencia de tamaño $\alpha$ (TUMP) sí y sólo sí
\begin{enumerate}
\item $\sup \beta_{C}(\theta)=\alpha$ $\theta \in \Theta_{0}$
\item Para cualquier otro test basado en $\left(X_{1}, \cdots X_{n}\right)$ con región crítica $C^{\prime}$ tal que $\sup _{\theta \in \Theta_{0}} \beta_{C^{\prime}}(\theta) \leq \alpha$, es $\beta_{C}(\theta) \geq \beta_{C^{\prime}}(\theta), \forall \theta \in \Theta_{1}$
\end{enumerate}
\end{proposición}

\subsection{Hipótesis nula simple frente a alternativa simple}

\begin{teorema} [Teorema de Neyman-Pearson - Parte I] 
Para contrastar $H_{0}: \theta=\theta_{0}$ frente a $H_{1}: \theta=\theta_{1}$, si para algún $k \geq 0$ existe un test con región crítica $c=\left\{\left(x_{1}, \cdots, x_{n}\right) \in \chi^{n}: \frac{f_{\theta_{1}}\left(x_{1}, \cdots, x_{n}\right)}{f_{0}\left(x_{1}, \cdots, x_{n}\right)} \geq k\right\}$ y región de aceptación $C^{c}=\left\{\left(x_{1}, \cdots, x_{n}\right) \in x^{n}: \frac{f_{\theta_{1}}\left(x_{1}, \cdots, x_{n}\right)}{f_{\theta_{0}}\left(x_{1}, \cdots x_{n}\right)}<k\right\}$ tal que $\alpha=P_{\theta_{0}}(C)$, entonces $C$ es uniformemente de máxima potencia de tamaño $\alpha$
\end{teorema}

\begin{proof}
Observemos que $C$ es un test de tamaño $\alpha$ ya que $\Theta_{0}=\left\{\theta_{0}\right\}$ y por lo tanto $\sup _{\theta \in \Theta_{0}} \beta_{C}(\theta)=P_{\theta_{0}}(C)=\alpha$\\
Sea $C^{\prime}$ otro test de nivel $\alpha$, es decir tal que $\alpha \geq \sup _{\theta \in \Theta_{0}} \beta_{C^{\prime}}(\theta)=P_{\theta_{0}}\left(C^{\prime}\right)$ y consideremos la siguiente partición del espacio muestral.
$$
\begin{gathered}
S^{+}=\left\{\left(x_{1}, \cdots x_{n}\right) \in \chi^{n}: \mathrm{I}_{C}\left(x_{1}, \cdots x_{n}\right)>\mathrm{I}_{C^{\prime}}\left(x_{1}, \cdots x_{n}\right)\right\}, \\
S^{-}=\left\{\left(x_{1}, \cdots x_{n}\right) \in \chi^{n}: \mathrm{I}_{C}\left(x_{1}, \cdots x_{n}\right)<\mathrm{I}_{C^{\prime}}\left(x_{1}, \cdots x_{n}\right)\right\}, \\
\chi^{n}-S^{+} \bigcup S^{-}=\left\{\left(x_{1}, \cdots x_{n}\right) \in \chi^{n}: \mathrm{I}_{C}\left(x_{1}, \cdots x_{n}\right)=\mathrm{I}_{C^{\prime}}\left(x_{1}, \cdots x_{n}\right)\right\} \\
\int_{\chi^{n}}\left(I_{C}\left(x_{1}, \cdots, x_{n}\right)-I_{C^{\prime}}\left(x_{1}, \cdots, x_{n}\right)\right)\left(f_{\theta_{1}}\left(x_{1}, \cdots, x_{n}\right)-k f_{\theta_{0}}\left(x_{1}, \cdots, x_{n}\right)\right) d x_{1} \cdots d x_{n}= \\
\int_{S^{+}}\left(I_{C}\left(x_{1}, \cdots, x_{n}\right)-I_{C^{\prime}}\left(x_{1}, \cdots, x_{n}\right)\right)\left(f_{\theta_{1}}\left(x_{1}, \cdots, x_{n}\right)-k f_{\theta_{0}}\left(x_{1}, \cdots, x_{n}\right)\right) d x_{1} \cdots d x_{n}+ \\
\int_{S^{-}}\left(I_{C}\left(x_{1}, \cdots, x_{n}\right)-I_{C^{\prime}}\left(x_{1}, \cdots, x_{n}\right)\right)\left(f_{\theta_{1}}\left(x_{1}, \cdots, x_{n}\right)-k f_{\theta_{0}}\left(x_{1}, \cdots, x_{n}\right)\right) d x_{1} \cdots d x_{n}+ \\
\int_{\chi^{n}-S^{+} \cup S^{-}}\left(I_{C}\left(x_{1}, \cdots, x_{n}\right)-I_{C^{\prime}}\left(x_{1}, \cdots, x_{n}\right)\right)\left(f_{\theta_{1}}\left(x_{1}, \cdots, x_{n}\right)-k f_{\theta_{0}}\left(x_{1}, \cdots, x_{n}\right)\right) d x_{1} \cdots d x_{n} \geq 0 \\
\int_{\chi^{n}} I_{C}\left(x_{1}, \cdots, x_{n}\right) f_{\theta_{1}}\left(x_{1}, \cdots, x_{n}\right)-\int_{\chi^{n}} I_{C^{\prime}}\left(x_{1}, \cdots, x_{n}\right) f_{\theta_{1}}\left(x_{1}, \cdots, x_{n}\right) d x_{1} \cdots d x_{n} \geq \\
k\left(\int_{\chi^{n}} I_{C}\left(x_{1}, \cdots, x_{n}\right) f_{\theta_{0}}\left(x_{1}, \cdots, x_{n}\right)-\int_{\chi^{n}} I_{C^{\prime}}\left(x_{1}, \cdots, x_{n}\right) f_{\theta_{0}}\left(x_{1}, \cdots, x_{n}\right) d x_{1} \cdots d x_{n}\right) \\
P_{\theta_{1}}(C)-P_{\theta_{1}}\left(C^{\prime}\right) \geq k\left(P_{\theta_{0}}(C)-P_{\theta_{0}}\left(C^{\prime}\right)\right) \geq k(\alpha-\alpha)=0 \Rightarrow \beta_{C}(\theta) \geq \beta C^{\prime}(\theta), \forall \theta \in \Theta_{1}=\left\{\theta_{1}\right\}
\end{gathered}
$$
\end{proof}

\begin{observación}
De la demostración del teorema se deduce que los puntos para los que $f\left(x_{1}, \ldots, x_{n} \mid \theta_{1}\right)=k f\left(x_{1}, \ldots, x_{n} \mid \theta_{0}\right)$ pueden ser colocados tanto en la región crítica como en la región de aceptación.\\
Es importante señalar que el teorema de Neyman-Pearson no dice que el test de la forma dada en su enunciado deba existir cualquiera que sea $\alpha \in[0,1]$
\end{observación}


\ejemplo{

Para una muestra de tamaño $n=12$, extraída de una distribución de Poisson con parámetro $\theta$, donde $\theta \in [0, 0.5]$, se plantea el siguiente contraste de hipótesis:

\[
\begin{cases}
H_{0}: \theta = 0 \\
H_{1}: \theta = 0.5
\end{cases}
\]

La región crítica para este contraste viene dada por:
\[
C = \{(x_1, \ldots, x_{12}) : \sum_{i=1}^{12} x_i \geq 2 \}
\]

En este caso particular, como $\sum_{i=1}^{12} x_i < 2$, se tiene que $\overline{X} < \frac{1}{6}$.

La probabilidad de error de tipo I ($\alpha$) y la función de potencia ($\beta(\theta)$) se calculan como sigue:

\begin{align*}
\alpha &= \beta(0) = P\left(C \mid \theta=0\right) = P\left(\sum_{i=1}^{12} x_i \geq 2 \mid \theta=0\right) = 0 \\
\beta(0.5) &= P\left(C \mid \theta=0.5\right) = P\left(\sum_{i=1}^{12} x_i \geq 2 \mid \theta=0.5\right) \\
&= P\left(\text{Poisson}(6) \geq 2\right) \\
&= 1 - P\left(\text{Poisson}(6) = 0\right) - P\left(\text{Poisson}(6) = 1\right) \\
&= 1 - e^{-6}\left(\frac{6^0}{0!} + \frac{6^1}{1!}\right) \\
&= 1 - e^{-6}(1 + 6) \approx 0.9826 \\
\beta(0.25) &= P\left(C \mid \theta=0.25\right) = P\left(\sum_{i=1}^{12} x_i \geq 2 \mid \theta=0.25\right) \\
&= P\left(\text{Poisson}(3) \geq 2\right) \\
&= 1 - P\left(\text{Poisson}(3) = 0\right) - P\left(\text{Poisson}(3) = 1\right) \\
&= 1 - e^{-3}\left(\frac{3^0}{0!} + \frac{3^1}{1!}\right) \\
&= 1 - e^{-3}(1 + 3) \approx 0.8009
\end{align*}

}

\ejemplo{
Para una m.a.s.(n) de $X \sim N(\theta, \sigma)$, con $\sigma$ conocida, encontrar el TUMP de tamaño $\alpha$ para contrastar $H_{0}: \theta=\theta_{0}$ frente a $H_{1}: \theta=\theta_{1}$, con $\theta_{0}<\theta_{1}$\\
$\frac{f_{\theta_{1}}\left(x_{1}, \cdots, x_{n}\right)}{f_{0}\left(x_{1}, \cdots x_{n}\right)}=e^{\frac{1}{\sigma^{2}} n\left(\theta_{0}^{2}-\theta_{1}^{2}\right)} e^{\frac{1}{\sigma^{n} n} n\left(\theta_{1}-\theta_{0}\right)} \geq k \Leftrightarrow \bar{x} \geq c$\\
donde $c$ es tal que $P_{\theta_{0}}\{\bar{x} \geq c\}=\alpha$, es decir $c=\theta_{0}+z_{\alpha} \frac{\sigma}{\sqrt{n}}$
}

\ejemplo{
Para una m.a.s.(n) de $X \sim \operatorname{Exp}(\theta)$, encontrar el TUMP de tamaño $\alpha$ para contrastar $H_{0}: \theta=\theta_{0}$ frente a $H_{1}: \theta=\theta_{1}$, con $\theta_{0}<\theta_{1}$\\
$\frac{f_{9}\left(x_{1}, \cdots, x_{n}\right)}{f_{\theta_{0}}\left(x_{1}, \cdots x_{n}\right)}=\left(\frac{\theta_{1}}{\theta_{0}}\right)^{n} e^{\left(\theta_{0}-\theta_{1}\right) \sum_{i=1}^{n} x_{i}} \geq k \Leftrightarrow 2 \theta_{0} \sum_{i=1}^{n} x_{i} \leq c$\\
donde $c$ es tal que $P_{\theta_{0}}\left\{2 \theta_{0} \sum_{i=1}^{n} x_{i} \leq c\right\}=\alpha$, es decir $c=\chi_{2 n, \alpha}^{2}$
}

\ejemplo{
Para una m.a.s.(n) de $X \sim N(\mu, \theta)$, con $\mu$ conocida, encontrar el TUMP de tamaño $\alpha$ para contrastar $H_{0}: \theta=\theta_{0}$ frente a $H_{1}: \theta=\theta_{1}$, con $\theta_{0}<\theta_{1}$\\
$\frac{f_{\theta_{1}}\left(x_{1}, \cdots, x_{n}\right)}{f_{\theta_{0}}\left(x_{1}, \cdots x_{n}\right)}=\left(\frac{\theta_{0}}{\theta_{1}}\right)^{n} e^{\frac{1}{2}\left(\frac{1}{\theta_{0}^{2}}-\frac{1}{\theta_{1}^{2}}\right) \sum_{i=1}^{n}\left(x_{i}-\mu\right)^{2}} \geq k \Leftrightarrow \frac{\sum_{i=1}^{n}\left(x_{i}-\mu\right)^{2}}{\theta_{0}^{2}} \geq c$ donde $c$ es tal que $P_{\theta_{0}}\left\{\frac{\sum_{i=1}^{n}\left(x_{i}-\mu\right)^{2}}{\theta_{0}^{2}} \geq c\right\}=\alpha$, es decir $c=\chi_{n, \alpha}^{2}$
}

\ejemplo{
Para una m.a.s.(n) de $X \sim \operatorname{Bin}(1, \theta)$, encontrar el TUMP de tamaño $\alpha$ para contrastar $H_{0}: \theta=\theta_{0}$ frente a $H_{1}: \theta=\theta_{1}$, con $\theta_{0}<\theta_{1}$\\
$\frac{f_{0}\left(x_{1}, \cdots, x_{n}\right)}{f_{\theta_{0}}\left(x_{1}, \cdots x_{n}\right)}=\left(\frac{1-\theta_{1}}{1-\theta_{0}}\right)^{n}\left(\frac{\theta_{1}}{\theta_{0}} \frac{1-\theta_{0}}{1-\theta_{1}}\right)^{\sum_{i=1}^{n} x_{i}} \Leftrightarrow \sum_{i=1}^{n} x_{i} \geq c$\\
donde $c$ es tal que $P_{\theta_{0}}\left\{\sum_{i=1}^{n} x_{i} \geq c\right\}=\sum_{j=c}^{n}\binom{n}{j} \theta_{0}^{j}\left(1-\theta_{0}\right)^{n-j}=\alpha_{c}, c=0,1,2, \ldots, n$
}


\begin{teorema} [Teorema de Neyman-Pearson - Parte II]
Para contrastar $H_{0}: \theta=\theta_{0}$ frente a $H_{1}: \theta=\theta_{1}$, si para algún $k>0$ existe un test con región crítica $C$ tal que $P_{\theta_{0}}(C)=\alpha$ con $\left\{\left(x_{1}, \cdots, x_{n}\right) \in \chi^{n}: \frac{f_{9}\left(x_{1}, \cdots, x_{n}\right)}{f_{0}\left(x_{1}, \cdots x_{n}\right)}>k\right\} \subset c c\left\{\left(x_{1}, \cdots, x_{n}\right) \in x^{n}: \frac{f_{9}\left(x_{1}, \cdots, x_{n}\right)}{f_{0}\left(x_{1}, \cdots x_{n}\right)} \geq k\right\}$ entonces cualquier test $C^{\prime}$ uniformemente de máxima potencia de nivel $\alpha$, es de tamaño $\alpha$ y verifica\\
$\left\{\left(x_{1}, \cdots, x_{n}\right) \in \chi^{n}: \frac{f_{\theta_{1}}\left(x_{1}, \cdots, x_{n}\right)}{f_{f_{0}}\left(x_{1}, \cdots x_{n}\right)}>k\right\} \subset c^{\prime} \subset\left\{\left(x_{1}, \cdots, x_{n}\right) \in x^{n}: \frac{f_{\theta_{1}}\left(x_{1}, \ldots, x_{n}\right)}{f_{\theta_{0}}\left(x_{1}, \cdots x_{n}\right)} \geq k\right\}$ salvo quizás en un conjunto $A \subset \chi^{n}$ tal que $P_{\theta_{0}}(A)=P_{\theta_{1}}(A)$
\end{teorema}

\begin{proof}
Si $C^{\prime}$ es un test uniformemente de máxima potencia de nivel $\alpha$ y existe $C$ de la forma del enunciado con $k>0$, entonces por el apartado anterior $C$ es también uniformemente de máxima potencia de nivel $\alpha$. Por lo tanto, $\beta_{C}\left(\theta_{1}\right)=\beta_{C^{\prime}}\left(\theta_{1}\right)$. Entonces, $0=P_{\theta_{1}}(C)-P_{\theta_{1}}\left(C^{\prime}\right) \geq k\left(\alpha-P_{\theta_{0}}\left(C^{\prime}\right)\right) \geq 0$ y como $k>0 \Rightarrow P_{\theta_{0}}\left(C^{\prime}\right)=\alpha$ y se sigue que

$$
\int_{\chi^{n}}\left(I_{C}\left(x_{1}, \cdots, x_{n}\right)-I_{C^{\prime}}\left(x_{1}, \cdots, x_{n}\right)\right)\left(f_{\theta_{1}}\left(x_{1}, \cdots, x_{n}\right)-k f_{\theta_{0}}\left(x_{1}, \cdots, x_{n}\right)\right) d x_{1} \cdots d x_{n}=0
$$
Por lo tanto, o bien $I_{C}\left(x_{1}, \cdots, x_{n}\right)=I_{C^{\prime}}\left(x_{1}, \cdots, x_{n}\right), \forall\left(x_{1}, \cdots, x_{n}\right)$, o

$$
\begin{gathered}
\int_{S^{+}}\left(I_{C}\left(x_{1}, \cdots, x_{n}\right)-I_{C^{\prime}}\left(x_{1}, \cdots, x_{n}\right)\right)\left(f_{\theta_{1}}\left(x_{1}, \cdots, x_{n}\right)-k f_{\theta_{0}}\left(x_{1}, \cdots, x_{n}\right)\right) d x_{1} \cdots d x_{n}=0 \\
\int_{S^{-}}\left(I_{C}\left(x_{1}, \cdots, x_{n}\right)-I_{C^{\prime}}\left(x_{1}, \cdots, x_{n}\right)\right)\left(f_{\theta_{1}}\left(x_{1}, \cdots, x_{n}\right)-k f_{\theta_{0}}\left(x_{1}, \cdots, x_{n}\right)\right) d x_{1} \cdots d x_{n}=0 \\
\int_{S^{+}}\left(f_{\theta_{1}}\left(x_{1}, \cdots, x_{n}\right)-k f_{\theta_{0}}\left(x_{1}, \cdots, x_{n}\right)\right) d x_{1} \cdots d x_{n}=0 \Rightarrow S^{+} \subset\left\{f_{\theta_{1}}\left(x_{1}, \cdots, x_{n}\right)-k f_{\theta_{0}}\left(x_{1}, \cdots, x_{n}\right)=0\right\} \\
-\int_{S^{-}}\left(f_{\theta_{1}}\left(x_{1}, \cdots, x_{n}\right)-k f_{\theta_{0}}\left(x_{1}, \cdots, x_{n}\right)\right) d x_{1} \cdots d x_{n}=0 \Rightarrow S^{-} \subset\left\{f_{\theta_{1}}\left(x_{1}, \cdots, x_{n}\right)-k f_{\theta_{0}}\left(x_{1}, \cdots, x_{n}\right)=0\right\} \\
\text { o bien } P_{\theta_{1}}\left(S^{+}\right)=\int_{S^{+}} f\left(x_{1}, \ldots, x_{n} \mid \theta_{1}\right) d x_{1} \cdots d x_{n}=0, P_{\theta_{0}}\left(S^{+}\right)=\int_{S^{+}} f\left(x_{1}, \ldots, x_{n} \mid \theta_{0}\right) d x_{1} \cdots d x_{n}=0 \\
P_{\theta_{1}}\left(S^{-}\right)=\int_{S^{-}} f\left(x_{1}, \ldots, x_{n} \mid \theta_{1}\right) d x_{1} \cdots d x_{n}=0 \text { y } P_{\theta_{1}}\left(S^{-}\right)=\int_{S^{-}} f\left(x_{1}, \ldots, x_{n} \mid \theta_{0}\right) d x_{1} \cdots d x_{n}=0
\end{gathered}
$$
\end{proof}

\begin{definición} [Test aleatorizado]
Un test aleatorizado es cualquier función medible tal que $\varphi\left(x_{1}, \cdots, x_{n}\right)$ expresa la probabilidad de rechazar la hipótesis nula cuando se observa $\left(x_{1}, \cdots, x_{n}\right) \in \chi^{n}$
\end{definición}

\begin{observación}
Como su propio nombre indica, en un test aleatorizado, observado un valor muestral $\left(x_{1}, \cdots, x_{n}\right) \in \chi^{n}$, se efecua un sorteo con probabilidad $\varphi\left(x_{1}, \cdots, x_{n}\right)$ de rechazar $H_{0}$ y $1-\varphi\left(x_{1}, \cdots, x_{n}\right)$ de aceptarla. En este sentido, una región crítica es un test no aleatorizado, pues observada una muestra nuestra decisión es tajante: rechazamos o aceptamos $H_{0}$.

En cambio, en los test aleatorizados la decisión final depede total o parcialmente del azar. Aunque esta es una regla de conducta no determinística, los tests no aleatorizados son un caso particular de ella para $\varphi\left(x_{1}, \cdots, x_{n}\right)=I_{C}\left(x_{1}, \cdots, x_{n}\right)$    
\end{observación}

\begin{definición} [Función de potencia de un test aleatorizado]
Si $\varphi$ es un test aleatorizado para el contraste $H_{0}: \theta \in \Theta_{0}$ frente a $H_{1}: \theta \in \Theta_{1}$, se define función de potencia del test a la función $\beta_{\varphi}: \Theta \rightarrow[0,1]$ que a cada valor $\theta$ del parámetro le asigna el valor $\beta_{\varphi}(\theta)=E_{\theta}(\varphi)$.
\end{definición} 

\begin{definición} [Nivel de significación y tamaño de un test aleatorizado]
Un test aleatorizado $\varphi$ tiene nivel de significación $\alpha \in[0,1]$ sí y sólo sí $\sup \beta_{\varphi}(\theta) \leq \alpha$ y se denomina tamaño del test al valor $\sup_{\theta \in \Theta_{0}}\beta_{\varphi}(\theta)$ $\theta \in \Theta_{0}$
\end{definición}

\begin{teorema}
En las mismas condiciones del teorema de Neyman-Pearson, $\forall \alpha \in(0,1)$ existe un test aleatorizado $\varphi$ de tamaño $\alpha$ de la forma

$$
\varphi\left(x_{1}, \cdots x_{n}\right)=\left\{\begin{array}{ccc}
1 & \text { si } & f_{\theta_{1}}\left(x_{1}, \cdots x_{n}\right)>k f_{\theta_{0}}\left(x_{1}, \cdots x_{n}\right) \\
\gamma & \text { si } & f_{\theta_{1}}\left(x_{1}, \cdots x_{n}\right)=k f_{\theta_{0}}\left(x_{1}, \cdots x_{n}\right) \\
0 & \text { si } & f_{\theta_{1}}\left(x_{1}, \cdots x_{n}\right)<k f_{\theta_{0}}\left(x_{1}, \cdots x_{n}\right)
\end{array}\right.
$$
con $k \geq 0$ y $\gamma \in[0,1]$ tales que
$$
\alpha=E_{\theta_{0}}[\varphi]=P_{\theta_{0}}\left\{f_{\theta_{1}}\left(x_{1}, \cdots x_{n}\right)>k f_{\theta_{0}}\left(x_{1}, \cdots x_{n}\right)\right\}+\gamma P_{\theta_{0}}\left\{f_{\theta_{1}}\left(x_{1}, \cdots x_{n}\right)=k f_{\theta_{0}}\left(x_{1}, \cdots x_{n}\right)\right\}
$$

Además,\\
I) $\varphi$ es uniformemente de máxima potencia de tamaño $\alpha$ para contrastar $H_{0}: \theta=\theta_{0}$ frente a $H_{1}: \theta=\theta_{1}$

Para cualquier otro test $\varphi^{\prime}$ basado en $\left(X_{1}, \cdots X_{n}\right)$ tal que $\sup _{\theta \in \Theta_{0}} \beta_{\varphi^{\prime}}(\theta) \leq \alpha$, es $\beta_{\varphi}(\theta) \geq \beta_{\varphi^{\prime}}(\theta), \forall \theta \in \Theta_{1}$\\
II) existe $\varphi$ de la forma del enunciado verificando $\alpha=E_{\theta_{0}}[\varphi]$ para $k>0$ y $\varphi^{\prime}$ es uniformemente de máxima potencia de nivel $\alpha$, entonces $\varphi^{\prime}$ es de tamaño $\alpha$ y $\varphi^{\prime}\left(x_{1}, \cdots, x_{n}\right)=\varphi\left(x_{1}, \cdots, x_{n}\right)$ salvo quizá en un conjunto $A \subset \chi^{n}$ tal que $P_{\theta_{0}}(A)=P_{\theta_{1}}(A)=0$    
\end{teorema}


\begin{definición} [Test insesgado]
Un test $\varphi$ de tamaño $\alpha$ es insesgado para el contraste $H_{0}: \theta=\theta_{0}$ frente a $H_{1}: \theta=\theta_{1}$ sí y sólo sí $E_{\theta_{1}}(\varphi) \geq \alpha$
\end{definición}

\begin{corolario}

El test uniformemente de máxima potencia de tamaño $\alpha$ construido en el lema de Neyman Pearson es insesgado.
\end{corolario}

\begin{proof}
Sea $\varphi^{\prime}\left(x_{1}, \cdots, x_{n}\right)=\alpha$, salvo quizá en un conjunto $A \subset \chi^{n}$ tal que $P_{\theta_{0}}(A)=P_{\theta_{1}}(A)=0$. Como $E_{\theta_{0}}\left[\varphi^{\prime}\right]=\alpha$ y $\varphi^{\prime}$ no es de la forma del enunciado del Teorema 5, $\alpha=E_{\theta_{1}}\left[\varphi^{\prime}\right] \leq E_{\theta_{1}}[\varphi]$.
\end{proof}

\ejemplo{
Para una m.a.s.(n) de $X \sim \operatorname{Bin}(1, \theta)$, encontrar el TUMP de tamaño $\alpha$ para contrastar $H_{0}: \theta=\theta_{0}$ frente a $H_{1}: \theta=\theta_{1}$, con $\theta_{0}>\theta_{1}$. Particularizarlo para $n=10, \alpha=0.05, \theta_{0}=0.5$ y $\theta_{1}=0.4$\\
$\frac{f_{\theta_{1}}\left(x_{1}, \cdots, x_{n}\right)}{f_{\theta_{0}}\left(x_{1}, \cdots x_{n}\right)}=\left(\frac{1-\theta_{1}}{1-\theta_{0}}\right)^{n}\left(\frac{\theta_{1}}{\theta_{0}} \frac{1-\theta_{0}}{1-\theta_{1}}\right)^{\sum_{i=1}^{n} x_{i}} \Leftrightarrow \sum_{i=1}^{n} x_{i} \leq c$\\
donde $c$ es tal que $P_{\theta_{0}}\left\{\sum_{i=1}^{n} x_{i} \leq c\right\}=\sum_{j=0}^{c}\binom{n}{j} \theta_{0}^{j}\left(1-\theta_{0}\right)^{n-j}=\alpha_{c}, c=0,1,2, \ldots, n$\\
$P_{\theta_{0}}\left\{\sum_{i=1}^{n} x_{i}<2\right\}=0.0108$\\
$P_{\theta_{0}}\left\{\sum_{i=1}^{n} x_{i}<3\right\}=0.0547$

$$
\varphi\left(x_{1}, \cdots x_{n}\right)=\left\{\begin{array}{lll}
1 & \text { si } & \sum_{i=1}^{n} x_{i}<2 \\
\gamma & \text { si } & \sum_{i=1}^{n=1} x_{i}=2 \\
0 & \text { si } & \sum_{i=1} x_{i}>2
\end{array}\right.
$$

$0.05=E_{\theta_{0}}[\varphi]=1 \times P_{\theta_{0}}\left\{\sum_{i=1}^{n} x_{i}<2\right\}+\gamma \times P_{\theta_{0}}\left\{\sum_{i=1}^{n} x_{i}=2\right\}=0.0108+\gamma \times 0.0547 \Rightarrow \gamma=0.892$
}



\section*{Contrastes de hipótesis unilaterales}
Familia de distribuciones de razón de verosimilitud monótona Sea $X \approx\left(\chi, \beta_{\chi}, F_{\theta}\right)_{\theta \in \Theta \subset \mathbb{R}}$ un modelo estadístico continuo (o discreto) uniparamétrico y $\left(X_{1}, \cdots, X_{n}\right)$ una muestra de $\left\{F_{\theta}, \theta \in \Theta\right\}$, siendo $f_{\theta}\left(x_{1}, \cdots x_{n}\right)$ su función de densidad (o de masa) $\left\{F_{\theta}, \theta \in \Theta\right\}$ es una familia de distribuciones de razón de verosimilitud monótona creciente (o decreciente) sí y sólo sí existe un estadístico $T=T\left(X_{1}, \cdots, X_{n}\right): \chi^{n} \rightarrow \mathbb{R}$ tal que, si $\theta_{0}, \theta_{1} \in \Theta$ y $\theta_{0}<\theta_{1}$, entonces la razón de verosimilitudes $\frac{f_{\theta_{1}}\left(x_{1}, \cdots, x_{n}\right)}{f_{\theta_{0}}\left(x_{1}, \cdots, x_{n}\right)}$ es una función monótona creciente (o decreciente) en $T\left(x_{1}, \cdots, x_{n}\right)$