\section{Estadística Descriptiva}

\subsection{Introducción}
El Cálculo de Probabilidades proporciona una Teoría Matemática que permite analizar las propiedades de los Experimentos Aleatorios\\
"La velocidad del movimiento caótico de las moléculas de un gas sigue una distribución normal de parámetros..."\\
"La vida de un determinado tipo de componente eléctrica tiene distribución exponencial de media..."

Construir un Espacio Probabilístico que sirva de Modelo Estadístico asociado a una determinada Variable Aleatoria real para la Deducción de Consecuencias

Para tratar de averiguar si una moneda está trucada no hay mejor procedimiento que lanzarla un buen número de veces y verificar si estadísticamente los resultados obtenidos confirman o invalidan la hipótesis $p=0.5$, siendo $p$ la probabilidad de cara. Desde el Cálculo de Probabilidades sólo se podrá actuar en función del parámetro p sin alcanzar soluciones numéricas

Disponer de un Conjunto de Observaciones del fenómeno considerado (en lugar de un espacio probabilístico totalmente especificado) hace abandonar los dominios del Cálculo de Probabilidades para introducirse en el terreno de la Estadística Matemática o Inferencia Estadística, cuya finalidad es obtener información sobre la Ley de Probabilidad de dicho fenómeno a partir del Análisis e Interpretación de las observaciones recolectadas

\begin{displayquote}
Estadística Descriptiva o Análisis de Datos: Recolección de la Información y su Tratamiento Numérico
\end{displayquote}

Métodos Estadísticos e Inferencia Estadística: Conjunto de Técnicas que utilizan la Información para construir Modelos Matemáticos en situaciones prácticas de incertidumbre y Análisis e Interpretación de las observaciones como método para obtener conclusiones sobre la Ley de Probabilidad del fenómeno en estudio

Inferencia Frecuentista e Inferencia Bayesiana\\
Modelos Estadísticos Paramétricos y No Paramétricos

Estimación Puntual: Pronóstico de un determinado parámetro de la distribución mediante un único valor numérico

Etimación por Intervalo: Intervalo numérico de valores en el que se pueda afirmar razonablemente que varía el parámetro en cuestión

Contraste de Hipótesis: Corroborar o Invalidar una determinada afirmación acerca de la distribución del fenómeno estudiado.

Concepto de Población y Muestra Aleatoria

