\begin{ejercicio}[Ejercicio 2.7.1]
    Sea $X$ una observación de una población $N(0, \sigma)$. ¿Es $|X|$ un estadístico suficiente?
\end{ejercicio}

\begin{solucion}
    Para una variable aleatoria $X$ con distribución normal $N(0, \sigma)$, la función de densidad de probabilidad es:
    \begin{equation}
        f(x|\sigma) = \frac{1}{\sqrt{2\pi}\sigma}e^{-\frac{|x|^2}{2\sigma^2}}
    \end{equation}
    Podemos entonces aplicar el teorema de factorización (2.2.1) para obtener una factorización de la siguiente forma:
    \begin{equation}
        f(x|\sigma) = h(x)g(|x|, \sigma)
    \end{equation}
    Donde nuestro estadístico es $T(X) = |X|$. Podemos tomar:
    \begin{align}
        h(x) &= \frac{1}{\sqrt{2\pi}} \\
        g(|x|, \sigma) &= \frac{1}{\sigma} e^{-\frac{|x|^2}{2\sigma^2}}
    \end{align}
    Finalmente, podemos ver que:
    \begin{equation}
        f(x|\sigma) = h(x)g(|x|, \sigma)
    \end{equation}
    Por lo que $|X|$ es un estadístico suficiente para $\sigma$.
\end{solucion}