\section{Contraste de Hipótesis}

\subsection{Principios básicos de un contraste de hipótesis}
Sea $X \approx\left(\chi, \beta_{\chi}, F_{\theta}\right)_{\theta \in \Theta \subset \mathbb{R}^{\ell}}$ modelo estadístico $\ell$-paramétrico y $\left(X_{1}, \cdots X_{n}\right)$ muestra de $\left\{F_{\theta}, \theta \in \Theta\right\}$

Idea: estudiar si una determinada afirmación sobre $\left\{F_{\theta}, \theta \in \Theta\right\}$ es confirmada o invalidada a partir de los datos muestrales

\ejemplo{
    Supongamos que en un laboratorio se está estudiando cierta reacción química sobre una determinada sustancia y que el resultado de dicha reacción es una variable observable que se puede modelizar mediante una v.a. $X$ con distribución normal. Por experiencias anteriores se sabe que, si en la sustancia está presente cierto mineral, $X \sim N(\mu=10, \sigma=4)$ y si no lo está $X \sim N(\mu=11, \sigma=4)$. Se puede comprobar por medio de unos análisis si el mineral está o no presente en la sustancia en estudio, pero dichos análisis son muy costosos, por lo que se procede a realizar la reacción química $n=25$ veces para decidir, a la luz de los resultados, si $\mu=10$ o $\mu=11$
}

\begin{definición} [Hiptótesis Estadística]
Una hipótesis estadística es cualquier afirmación acerca de un modelo estadístico.
\end{definición}


\begin{definición} [Hipótesis Estadística Simple y Compuesta]
    Una hipótesis estadística es simple si especifica totalmente el modelo estadístico, en otro caso, se dice que es compuesta
\end{definición}

\begin{definición} [Hipótesis Estadística Nula y Alternativa]
Se llama hipótesis nula $H_{0}$ a la hipótesis de trabajo y es la hipótesis estadística que vamos a aceptar si no hay suficiente evidencia a partir de los datos para rechazarla, consecuentemente se llama hipótesis alternativa $H_{1}$ a la hipótesis estadística que se acepta si hay suficiente evidencia a partir de los datos para rechazar $H_{0}$
\end{definición}

\begin{definición} [Contraste de Hipótesis Paramétrico]
Un contraste de hipótesis paramétrico es una partición del espacio paramétrico $\Theta$ en dos subconjuntos $\Theta_{0}$ y $\Theta_{1}$ tales que $\Theta=\Theta_{0} \bigcup \Theta_{1}$ y $\Theta_{0} \bigcap \Theta_{1}=\phi$
\end{definición}

En el ejemplo anterior $\Theta=\left\{\mu_{0}, \mu_{1}\right\}, \Theta_{0}=\left\{\mu_{0}\right\}, \Theta_{1}=\left\{\mu_{1}\right\}$

En un problema de contraste de hipótesis paramétrico se pretende contrastar $H_{0}: \theta \in \Theta_{0}$ frente a $H_{1}: \theta \in \Theta_{1}$. Por supuesto, la decisión debe basarse en la evidencia aportada por la observación de una muestra $\left(X_{1}, \cdots X_{n}\right)$, o equivalentemente por la observación de un cierto estadístico $T=T\left(X_{1}, \cdots, X_{n}\right)$, que se denomina estadístico del contraste, y que será usualmente un estimador suficiente del parámetro $\theta$

El contraste entre dos hipótesis basado en un estadístico, exige conocer la distribución en el muestreo de dicho estadístico, para los diversos valores del parámetro. De hecho, la idea del contraste consiste en localizar un suceso que sea muy improbable cuando la hipótesis nula es cierta. Si, una vez observada la muestra, acontece dicho suceso, o bien es que el azar ha jugado la mala pasada de elegir una muestra "muy rara" o, como parece más razonable, la hipótesis nula era falsa

\begin{definición} [Región Crítica]
Sea una partición del espacio muestral $\chi^{n}$ en dos subconjuntos $C$ y $C^{c}$ tales que $\chi^{n}=C \bigcup C^{c}$ y $C \bigcap C^{c}=\phi . C$ es una región crítica para el contraste $H_{0}: \theta \in \Theta_{0}$ frente a $H_{1}: \theta \in \Theta_{1}$ sí y sólo sí, se rechaza $H_{0}$ cuando se observa un valor muestral $\left(x_{1}, \cdots, x_{n}\right) \in C$, en cuyo caso se acepta $H_{1}$. Consecuentemente, $C^{*}$ se denomina región de aceptación y si $\left(x_{1}, \cdots, x_{n}\right) \in C^{c}$, se dice que no hay suficiente evidencia estadística para rechazar $H_{0}$, en este sentido se acepta $H_{0}$
\end{definición}


Si $T=T\left(X_{1}, \cdots, X_{n}\right): \chi^{n} \rightarrow \tau$ es el estadístico del contraste, sea una partición de $\tau$ en dos subconjuntos $C_{\tau}$ y $C_{\tau}^{*}$ tales que $\tau=C_{\tau} \bigcup C_{\tau}^{*}$ y $C_{\tau} \bigcap C_{\tau}^{*}=\phi . C_{\tau}$ es una región crítica para el contraste $H_{0}: \theta \in \Theta_{0}$ frente a $H_{1}: \theta \in \Theta_{1}$ sí y sólo sí, se rechaza $H_{0}$ cuando se observa un valor muestral $\left(x_{1}, \cdots, x_{n}\right)$ tal que $T\left(x_{1}, \cdots, x_{n}\right) \in C_{\tau}$, en cuyo caso acepto $H_{1}$. Consecuentemente, $C_{\tau}^{*}$ se denomina región de aceptación\\
Así, $C=\left\{\left(x_{1}, \cdots, x_{n}\right) \in \chi^{n}: T\left(x_{1}, \cdots, x_{n}\right) \in C_{\tau}\right\}$\\
En el ejemplo anterior, podemos considerar como región crítica $C=\left\{\left(x_{1}, \cdots, x_{n}\right): \bar{x} \geq k\right\}$

En un problema de contraste no sólo es importante conocer las probabilidades de cada resultado posible, sino también valorar el riesgo que estamos dispuestos a correr al tomar una decisión equivocada. En el ejemplo anterior, al rechazar que en la sustancia está presente el mineral cuando en realidad si lo está, o bien al aceptar que en la sustancia está presente el mineral cuando en realidad no lo está. Ambos errores tienen consecuencias prácticas distintas

\subsection{Errores de tipo I y de tipo II}
Error de tipo I es el error que se comete cuando se rechaza $H_{0}$ siendo cierta. Error de tipo II es el error que se comete cuando se acepta $H_{1}$ siendo falsa

En el ejemplo anterior, las probabilidades de cometer error de tipo I y error de tipo II son $P(\mathrm{I})=P(\bar{x} \geq k \mid \mu=10)$ y $P(\mathrm{II})=P(\bar{x}<k \mid \mu=11)$

Lo idóneo sería conseguir un test para el que ambas probabilidades de error fuesen pequeñas. Sin embargo (salvo en casos excepcionales y triviales), la reducción de la probabilidad de error de tipo II se hace a costa del aumento de la probabilidad de error de tipo l y viceversa, con lo que el único procedimiento para disminuir ambas probabilidades de error simultáneamente es aumentar el tamaño muestral, lo que conlleva un incremento del coste del procedimiento que en la práctica puede ser prohibitivo

\begin{definición} [Función de Potencia]
Si $C$ es una región crítica para el contraste $H_{0}: \theta \in \Theta_{0}$ frente a $H_{1}: \theta \in \Theta_{1}$, se define la función de potencia del test como la función $\beta_{C}: \Theta \rightarrow[0,1]$ que a cada valor $\theta$ del parámetro le asigna el valor $\beta_{C}(\theta)=P_{\theta}(C)$, es decir, la probabilidad de rechazar $H_{0}$ cuando el valor del parámetro es $\theta$
\end{definición}

\begin{definición} [Nivel de significación y tamaño del test ]
Un test $C$ tiene nivel de significación $\alpha \in[0,1]$ sí y sólo sí $\sup \beta_{C}(\theta) \leq \alpha$ y se denomina tamaño del test al valor $\sup \beta_{C}(\theta)$ $\theta \in \Theta_{0}$
\end{definición}

\ejemplo{
En el ejemplo anterior, con $n=25, C=\{10<\bar{x}<10,006\}$, $P(I)=\beta_{C}(\mu=10)=P(10<\bar{x}<10.006 \mid \mu=10)=0.05$ $P(\mathrm{II})=1-\beta_{C}(\mu=11)=P(10<\bar{x}<10.006 \mid \mu=11)=0.976$

En el ejemplo anterior, con $n=25, C=\{\bar{x} \geq k\}$ y $\alpha=0.05$, $P(\mathrm{I})=\beta_{C}(\mu=10)=P(\bar{x} \geq 11.316 \mid \mu=10)=0.05$ $P(\mathrm{II})=1-\beta_{C}(\mu=11)=P(\bar{x}<11.316 \mid \mu=11)=0.6554$

En el ejemplo anterior, con $n=100, C=\{\bar{x} \geq k\}$ y $\alpha=0.05$,\\
$P(\mathrm{I})=\beta_{C}(\mu=10)=P(\bar{x} \geq 10.658 \mid \mu=10)=0.05$\\
$P(\mathrm{II})=1-\beta_{C}(\mu=11)=P(\bar{x}<10.658 \mid \mu=11)=0.196$
    
}

\begin{definición} [p-valor]
 Si para contrastar $H_{0}: \theta \in \Theta_{0}$ frente a $H_{1}: \theta \in \Theta_{1}$, el test tiene región crítica 
 $$C=\left\{\left(x_{1}, \ldots, x_{n}\right): T\left(x_{1}, \ldots, x_{n}\right) \geq k\right\}$$
 para $T$ un estadístico conveniente, y se observa la muestra $\left(x_{1}, \ldots, x_{n}\right)$, se denomina p -valor correspondiente a $\left(x_{1}, \ldots, x_{n}\right)$ al valor

$$
p\left(x_{1}, \ldots, x_{n}\right)=\sup _{\theta \in \Theta_{0}} P\left\{T\left(X_{1}, \ldots, X_{n}\right) \geq T\left(x_{1}, \ldots, x_{n}\right) \mid \theta\right\}
$$

Si el tamaño del test es $\alpha$ y observada la muestra $\left(x_{1}, \ldots, x_{n}\right)$ el p-valor correspondiente $p\left(x_{1}, \ldots, x_{n}\right) \leq \alpha$, entonces $\left(x_{1}, \ldots, x_{n}\right)$ pertenece a la región crítica y por lo tanto se rechaza $H_{0}$. Si el p -valor $p\left(x_{1}, \ldots, x_{n}\right)>\alpha$, entonces $\left(x_{1}, \ldots, x_{n}\right)$ pertenece a la región de aceptación y por lo tanto no hay suficiente evidencia estadística para rechazar $H_{0}$
\end{definición}

\ejemplo{
En el ejemplo anterior, con $n=100, C=\{\bar{x} \geq k\}$ y $\alpha=0.05$,\\
$P(I)=\beta_{c}(\mu=10)=P(\bar{x} \geq 10.658 \mid \mu=10)=0.05$\\
$P(\mathrm{II})=1-\beta_{C}(\mu=11)=P(\bar{x}<10.658 \mid \mu=11)=0.196$\\
Entonces, observada $\bar{x}=11$

$$
p(11)=P(\bar{X} \geq 11 \mid \mu=10)=P(Z \geq 2.5)=0.00621
$$

Por lo tanto, se rechaza $H_{0}: \mu=10$ a favor de $H_{1}: \mu=11$    
}


\begin{observación}
\vspace{-2.5em}
\begin{enumerate}
\item n y $\alpha$ son valores fijados de antemano
\item Las hipótesis nula y alternativa no son intercambiables puesto que el tratamiento que reciben es asimétrico, la asimetría queda matizada por el valor $\alpha$ que se elija como nivel de significación y por la probabilidad de error de tipo II que resulte una vez diseñado el test, pues podría ocurrir que para $\theta \in \Theta_{1}, P_{\theta}\left(C^{*}\right)=1-\beta_{C}(\theta) \leq \alpha$
\item  En el contraste de hipótesis planteado se considera $H_{0}$ como la hipótesis de interés, en el sentido que para poder invalidarla es necesario esgrimir una gran evidencia. Por consiguiente, los test de hipótesis se emplean con un carácter conservador, a favor de la hipótesis nula, ya que el nivel de significación que se fija, intenta garantizar que sea muy infrecuente rechazar una hipótesis nula correcta, y la preocupación por dejar vigente una hipótesis nula falsa es menor, pudiéndose aceptar en este último caso riesgos más altos. En este sentido, si el resultado de un contraste de hipótesis es aceptar $H_{0}$, debe interpretarse que las observaciones no han aportado suficiente evidencia para descartarla; mientras que, si se rechaza, es porque se está razonablemente seguro de que $H_{0}$ es falsa y, por consiguiente, aceptamos $H_{1}$
\end{enumerate}
\end{observación}

\begin{proposición} [Criterio de comparación de contrastes]
Si $C$ y $C^{\prime}$ son dos test con nivel de significación $\alpha$ basados en una muestra $\left(X_{1}, \cdots X_{n}\right)$ de $\left\{F_{\theta}, \theta \in \Theta\right\}$, para contrastar $H_{0}: \theta \in \Theta_{0}$ frente a $H_{1}: \theta \in \Theta_{1}$, tales que $\beta_{C}(\theta) \geq \beta_{C^{\prime}}(\theta), \forall \theta \in \Theta_{1}$, entonces $C$ es uniformemente más potente que $C^{\prime}$    
\end{proposición}



\subsection{Test uniformemente más potente de tamaño $\alpha$}


\begin{proposición}
Sea $C$ una región crítica para el contraste $H_{0}: \theta \in \Theta_{0}$ frente a $H_{1}: \theta \in \Theta_{1}$, basada en una muestra ( $X_{1}, \cdots X_{n}$ ) de $\left\{F_{\theta}, \theta \in \Theta\right\}$ $C$ es un test uniformemente de máxima potencia de tamaño $\alpha$ (TUMP) sí y sólo sí
\begin{enumerate}
\item $\sup \beta_{C}(\theta)=\alpha$ $\theta \in \Theta_{0}$
\item Para cualquier otro test basado en $\left(X_{1}, \cdots X_{n}\right)$ con región crítica $C^{\prime}$ tal que $\sup _{\theta \in \Theta_{0}} \beta_{C^{\prime}}(\theta) \leq \alpha$, es $\beta_{C}(\theta) \geq \beta_{C^{\prime}}(\theta), \forall \theta \in \Theta_{1}$
\end{enumerate}
\end{proposición}

\subsection{Hipótesis nula simple frente a alternativa simple}

\begin{teorema} [Teorema de Neyman-Pearson (Parte I)] 
Para contrastar $H_{0}: \theta=\theta_{0}$ frente a $H_{1}: \theta=\theta_{1}$, si para algún $k \geq 0$ existe un test con región crítica $c=\left\{\left(x_{1}, \cdots, x_{n}\right) \in \chi^{n}: \frac{f_{\theta_{1}}\left(x_{1}, \cdots, x_{n}\right)}{f_{0}\left(x_{1}, \cdots, x_{n}\right)} \geq k\right\}$ y región de aceptación $c^{*}=\left\{\left(x_{1}, \cdots, x_{n}\right) \in x^{n}: \frac{f_{\theta_{1}}\left(x_{1}, \cdots, x_{n}\right)}{f_{\theta_{0}}\left(x_{1}, \cdots x_{n}\right)}<k\right\}$ tal que $\alpha=P_{\theta_{0}}(C)$, entonces $C$ es uniformemente de máxima potencia de tamaño $\alpha$
\end{teorema}

\begin{proof}
Observemos que $C$ es un test de tamaño $\alpha$ ya que $\Theta_{0}=\left\{\theta_{0}\right\}$ y por lo tanto $\sup _{\theta \in \Theta_{0}} \beta_{C}(\theta)=P_{\theta_{0}}(C)=\alpha$\\
Sea $C^{\prime}$ otro test de nivel $\alpha$, es decir tal que $\alpha \geq \sup _{\theta \in \Theta_{0}} \beta_{C^{\prime}}(\theta)=P_{\theta_{0}}\left(C^{\prime}\right)$ y consideremos la siguiente partición del espacio muestral.
$$
\begin{gathered}
S^{+}=\left\{\left(x_{1}, \cdots x_{n}\right) \in \chi^{n}: \mathrm{I}_{C}\left(x_{1}, \cdots x_{n}\right)>\mathrm{I}_{C^{\prime}}\left(x_{1}, \cdots x_{n}\right)\right\}, \\
S^{-}=\left\{\left(x_{1}, \cdots x_{n}\right) \in \chi^{n}: \mathrm{I}_{C}\left(x_{1}, \cdots x_{n}\right)<\mathrm{I}_{C^{\prime}}\left(x_{1}, \cdots x_{n}\right)\right\}, \\
\chi^{n}-S^{+} \bigcup S^{-}=\left\{\left(x_{1}, \cdots x_{n}\right) \in \chi^{n}: \mathrm{I}_{C}\left(x_{1}, \cdots x_{n}\right)=\mathrm{I}_{C^{\prime}}\left(x_{1}, \cdots x_{n}\right)\right\} \\
\int_{\chi^{n}}\left(I_{C}\left(x_{1}, \cdots, x_{n}\right)-I_{C^{\prime}}\left(x_{1}, \cdots, x_{n}\right)\right)\left(f_{\theta_{1}}\left(x_{1}, \cdots, x_{n}\right)-k f_{\theta_{0}}\left(x_{1}, \cdots, x_{n}\right)\right) d x_{1} \cdots d x_{n}= \\
\int_{S^{+}}\left(I_{C}\left(x_{1}, \cdots, x_{n}\right)-I_{C^{\prime}}\left(x_{1}, \cdots, x_{n}\right)\right)\left(f_{\theta_{1}}\left(x_{1}, \cdots, x_{n}\right)-k f_{\theta_{0}}\left(x_{1}, \cdots, x_{n}\right)\right) d x_{1} \cdots d x_{n}+ \\
\int_{S^{-}}\left(I_{C}\left(x_{1}, \cdots, x_{n}\right)-I_{C^{\prime}}\left(x_{1}, \cdots, x_{n}\right)\right)\left(f_{\theta_{1}}\left(x_{1}, \cdots, x_{n}\right)-k f_{\theta_{0}}\left(x_{1}, \cdots, x_{n}\right)\right) d x_{1} \cdots d x_{n}+ \\
\int_{\chi^{n}-S^{+} \cup S^{-}}\left(I_{C}\left(x_{1}, \cdots, x_{n}\right)-I_{C^{\prime}}\left(x_{1}, \cdots, x_{n}\right)\right)\left(f_{\theta_{1}}\left(x_{1}, \cdots, x_{n}\right)-k f_{\theta_{0}}\left(x_{1}, \cdots, x_{n}\right)\right) d x_{1} \cdots d x_{n} \geq 0 \\
\int_{\chi^{n}} I_{C}\left(x_{1}, \cdots, x_{n}\right) f_{\theta_{1}}\left(x_{1}, \cdots, x_{n}\right)-\int_{\chi^{n}} I_{C^{\prime}}\left(x_{1}, \cdots, x_{n}\right) f_{\theta_{1}}\left(x_{1}, \cdots, x_{n}\right) d x_{1} \cdots d x_{n} \geq \\
k\left(\int_{\chi^{n}} I_{C}\left(x_{1}, \cdots, x_{n}\right) f_{\theta_{0}}\left(x_{1}, \cdots, x_{n}\right)-\int_{\chi^{n}} I_{C^{\prime}}\left(x_{1}, \cdots, x_{n}\right) f_{\theta_{0}}\left(x_{1}, \cdots, x_{n}\right) d x_{1} \cdots d x_{n}\right) \\
P_{\theta_{1}}(C)-P_{\theta_{1}}\left(C^{\prime}\right) \geq k\left(P_{\theta_{0}}(C)-P_{\theta_{0}}\left(C^{\prime}\right)\right) \geq k(\alpha-\alpha)=0 \Rightarrow \beta_{C}(\theta) \geq \beta C^{\prime}(\theta), \forall \theta \in \Theta_{1}=\left\{\theta_{1}\right\}
\end{gathered}
$$
\end{proof}

\begin{observación}
De la demostración del teorema se deduce que los puntos para los que $f\left(x_{1}, \ldots, x_{n} \mid \theta_{1}\right)=k f\left(x_{1}, \ldots, x_{n} \mid \theta_{0}\right)$ pueden ser colocados tanto en la región crítica como en la región de aceptación.\\
Es importante señalar que el teorema de Neyman-Pearson no dice que el test de la forma dada en su enunciado deba existir cualquiera que sea $\alpha \in[0,1]$
\end{observación}


\ejemplo{

Para una muestra de tamaño $n=12$, extraída de una distribución de Poisson con parámetro $\theta$, donde $\theta \in [0, 0.5]$, se plantea el siguiente contraste de hipótesis:

\[
\begin{cases}
H_{0}: \theta = 0 \\
H_{1}: \theta = 0.5
\end{cases}
\]

La región crítica para este contraste viene dada por:
\[
C = \{(x_1, \ldots, x_{12}) : \sum_{i=1}^{12} x_i \geq 2 \}
\]

En este caso particular, como $\sum_{i=1}^{12} x_i < 2$, se tiene que $\overline{X} < \frac{1}{6}$.

La probabilidad de error de tipo I ($\alpha$) y la función de potencia ($\beta(\theta)$) se calculan como sigue:

\begin{align*}
\alpha &= \beta(0) = P\left(C \mid \theta=0\right) = P\left(\sum_{i=1}^{12} x_i \geq 2 \mid \theta=0\right) = 0 \\
\beta(0.5) &= P\left(C \mid \theta=0.5\right) = P\left(\sum_{i=1}^{12} x_i \geq 2 \mid \theta=0.5\right) \\
&= P\left(\text{Poisson}(6) \geq 2\right) \\
&= 1 - P\left(\text{Poisson}(6) = 0\right) - P\left(\text{Poisson}(6) = 1\right) \\
&= 1 - e^{-6}\left(\frac{6^0}{0!} + \frac{6^1}{1!}\right) \\
&= 1 - e^{-6}(1 + 6) \approx 0.9826 \\
\beta(0.25) &= P\left(C \mid \theta=0.25\right) = P\left(\sum_{i=1}^{12} x_i \geq 2 \mid \theta=0.25\right) \\
&= P\left(\text{Poisson}(3) \geq 2\right) \\
&= 1 - P\left(\text{Poisson}(3) = 0\right) - P\left(\text{Poisson}(3) = 1\right) \\
&= 1 - e^{-3}\left(\frac{3^0}{0!} + \frac{3^1}{1!}\right) \\
&= 1 - e^{-3}(1 + 3) \approx 0.8009
\end{align*}

}