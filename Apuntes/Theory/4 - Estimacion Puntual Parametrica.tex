\section{Estimación Puntual Paramétrica}

\begin{definición}[Estimador]
  Sea $(\Omega, \mathcal{A}, \mathcal{P})$ el espacio probabilístico asociado a un experimento aleatorio $\mathcal{E}$, una variable aleatoria observable $X: \Omega \longrightarrow \mathbb{R}$ y su modelo estadístico asociado $\left(\chi, \mathcal{B}, F_{\theta}\right)_{\theta \in \Theta \subset \mathbb{R}^{\ell}}, \mathcal{B}=\mathcal{B}(\mathbb{R})$, y $\left(X_{1}, \cdots X_{n}\right)$ m.a.s. $(n) \sim X$
  \\Un estimador del parámetro $\theta$ es un estadístico $T=T\left(X_{1}, \cdots X_{n}\right): \chi^{n} \longrightarrow \Theta$ que se utiliza para determinar el valor desconocido $\theta$
\end{definición}

\begin{definición}[Estimador centrado o insesgado]
  Dado un estimador $T = T(X_1, \ldots X_n): \chi^{n} \to \Theta$ se dice que es centrado para $\theta$ cuando $E_{\theta}[T] = \theta$. \\
  Se dice asintóticamente insesgado o centrado cuando $\lim_{n \to \infty}E_{\theta}[T] = \theta$
\end{definición}

\begin{definición}[Sesgo]
  Se llama sesgo de un estimador a la diferencia $b(T, \theta) = E_{\theta}[T] - \theta$
\end{definición}

\ejemplo{
  Veamos que el estadístico $T = (\bar{X}, S^2)$ es un estimador centrado de $\theta = (\mu, \sigma^2):$\\
  $$E[\bar{X}] = E\left[\frac{1}{n}\sum_{i  = 1}^{n} X_i \right] = \frac{1}{n} \sum_{i = 1}^{n}E[X_i] = \frac{1}{n} \sum_{i = 1}^{n} \mu = \mu$$
  $$E[S^2] = E\left[\frac{1}{n-1}\sum_{i  = 1}^{n} (X_i - \bar{X})^2 \right] = \frac{1}{n-1} \sum_{i = 1}^{n}E[(X_i - \bar{X})^2] = \frac{1}{n-1} \sum_{i = 1}^{n} \sigma^2 = \sigma^2$$
}

\ejemplo{
  Demuestra que el estadístico $\sigma_n^2 = b_2 = \frac{1}{n} \sum_{i = 1}^{n}(X_i - \bar{X})^2$ es un estimador centrado de $h(\theta) = \frac{n-1}{n}\sigma^2$ y $b(\sigma_n^2, \sigma^2) = -\frac{\sigma^2}{n}$
  $$E[\sigma_n^2] = E\left[\frac{1}{n} \sum_{i = 1}^{n}(X_i - \bar{X})^2\right] = \frac{1}{n} \sum_{i = 1}^{n}E[(X_i - \bar{X})^2] = $$ $$ = \frac{1}{n} \sum_{i = 1}^{n}E[X_i^2 - 2X_i\bar{X} + \bar{X}^2] = \frac{1}{n} \sum_{i = 1}^{n}E[X_i^2] - 2E[X_i\bar{X}] + E[\bar{X}^2] = $$
  $$ = \frac{1}{n} \sum_{i = 1}^{n} E[X_i]^2 - \frac{2}{n}\sum_{i = 1}^{n}E[X_i\bar{X}] + \frac{1}{n}\sum_{i = 1}^{n}E[\bar{X}^2] = $$
  Sabemos que: $
  \begin{cases}
    Var(\bar{X}) = E[\bar{X}^2] - E[\bar{X}]^2 \iff \frac{\sigma^2}{n} = E[\bar{X}^2] - \mu^2 \iff E[\bar{X}^2] = \frac{\sigma^2}{n} + \mu^2\\
    Var(X_i) = E[X_i^2] - E[X_i]^2 \iff \sigma^2 = E[X_i^2] - \mu^2 \iff E[X_i^2] = \sigma^2 + \mu^2\\
  \end{cases} $
  $$ = \frac{1}{n} n(\sigma^2 + \mu ^2) + \frac{1}{n}n(\frac{\sigma^2}{n} + \mu^2) - \frac{2}{n}\sum_{i = 1}^{n}E[X_i \bar{X}] = $$
  Ahora desarrollemos el término que falta: 
  $$E[X_i \bar{X}] = E[X_i \frac{1}{n}\sum_{j = 1}^{n}X_j] = \frac{1}{n}\sum_{j = 1}^{n}E[X_iX_j] = \frac{1}{n}\sum_{j = 1\\ j \neq i}^{n}E[X_iX_j] + \frac{1}{n}E[X_i^2] = $$
  $$ = \frac{1}{n}\sum_{j = 1 \\ j \neq i}^{n}E[X_i]E[X_j] + \frac{1}{n}E[X_i^2] = \frac{1}{n}(n-1)\mu^2 + \frac{1}{n}(\sigma^2 + \mu^2) = \mu^2 + \frac{\sigma^2}{n}$$
  $$ \implies = 2\mu^2 + \sigma^2(1 + \frac{1}{n}) - \frac{2}{n}\left(\sum_{i = 1}^{n}\mu^2 + \frac{\sigma^2}{n}\right) = 2\mu^2 + \sigma^2(1 + \frac{1}{n}) - 2(\mu^2 + \frac{\sigma^2}{n}) = \sigma^2(1 - \frac{1}{n}) \implies$$
  $$ \implies E[\sigma_n^2] = \sigma^2(1 - \frac{1}{n}) = h(\theta) \text{ y } b(\sigma_n^2, \sigma^2) =  E[\sigma_n^2] - \sigma^2 = -\frac{\sigma^2}{n}$$
  $$ \implies E[\sigma_n^2] = \sigma^2\left(1 - \frac{1}{n}\right) = h(\theta) \text{ y } b(\sigma_n^2, \sigma^2) =  E[\sigma_n^2] - \sigma^2 = -\frac{\sigma^2}{n}$$
}

\begin{observación}
  \vspace{-\topsep} % Removes the default vertical spacing
\vspace{-\topsep} % Removes the default vertical spacing
\vspace{-\topsep} % Removes the default vertical spacing
  \begin{itemize}
    \item Puede ocurrir que no exista un estimador centrado de $\theta$
    \item Si $T$ es un estimador centrado para $\theta$, en general $h(T)$ no tiene por qué ser centrado para $h(\theta)$
    \item A pesar de que exista un estimador centrado para $\theta$, puede ser que no tenga sentido
  \end{itemize}
\end{observación}

\ejemplo{
  Sea una m.a.s. de tamaño $n = 1$ de una población que sigue una distribución $Bin(1, \theta)$ demostrar que $T(X) = X^2$ no es un estimador centrado de $\theta^2$:\\
  $$ X \sim Bin(1, \theta) \equiv Bernouilli(\theta) \implies X^2 \sim Bernouilli(\theta) \implies E[X^2] = \theta \neq \theta^2$$
}

\ejemplo{
  Sea una m.a.s. de tamaño $n = 1$ de una población que sigue una distribución $Bin(1, \theta)$ demostrar que no existe un estimador centrado de $\theta^2$:\\
  Sea $g$ estadístico tal que $E[g(X)] = \theta^2 \implies$
  NO LO SÉ HACER
}


\ejemplo{
  Sea una m.a.s. de tamaño $n = 1$ que sigue una distribución $X \sim Poisson(\theta)$ demuestra que $T(X) = (-2)^x$ es un estimador centrado para $h(\theta) = e^{-3\theta}$, pero $Var_{\theta}(T) = e^{4\theta} - e^{-6\theta} \to \infty$ no es un estimador de $h(\theta)$: 
  $$X \sim Poisson(\theta) \implies f_{\theta}(x) = \frac{e^{-\theta}\theta^x}{x!} \implies E[T] = E[(-2)^x] = \sum_{x = 0}^{+\infty} (-2)^x \frac{e^{-\theta}\theta^x}{x!} = \sum_{x = 0}^{+\infty} \frac{(-2\theta)^x}{x!}e^{-\theta} = $$
  $$e^{-\theta} \sum_{x = 0}^{+\infty} \frac{(-2\theta)^x}{x!} = e^{-\theta}e^{-2\theta} = e^{-3\theta} = h(\theta)$$
  $$Var_{\theta}(T) = E[T^2] - E[T]^2 = E[(-2)^{2x}] - e^{-6\theta} = \sum_{x = 0}^{n} (-2)^{2x} \frac{e^{-\theta}\theta^x}{x!} - e^{-6\theta} = \sum_{x = 0}^{n} \frac{(4\theta)^x}{x!}e^{-\theta} - e^{-6\theta} = $$
  $$e^{-\theta}\sum_{x = 0}^{n} \frac{(4\theta)^x}{x!} - e^{-6\theta} = e^{-\theta}e^{4\theta} - e^{-6\theta} = e^{3\theta} - e^{-6\theta} \to \infty$$
  Este procedimiento ha demostrador que $T(X)$ es centrado para $h(\theta)$, pero su varianza es demasiado grande (infinita) por lo que no es adecuado para la estimación. 
}

\ejemplo{
  Sea $(T_j)_{j \in \mathbb{N}}$ sucesión de estimadores centrados para $\theta$, entonces $\bar{T_k} = \frac{1}{k}\sum_{j = 1}^{k}T_j$ es un estimador centrado para $\theta$:
  $$E[\bar{T_k}] = E[\frac{1}{k}\sum_{j = 1}^{k}T_j] = \frac{1}{k}\sum_{j = 1}^{k}E[T_j] = \frac{1}{k}\sum_{j = 1}^{k}\theta = \theta$$
}

\begin{definición}[Estimadores consistentes]
  Una sucesión de estimadores $T_{n} = T(X_1, \ldots, X_n)$ es una sucesión de estimadores tales que $\forall \theta \in \Theta, E_{\theta}[T_n] \underset{n \to \infty}{\longrightarrow} \theta$ y $V_{\theta}(T_n) \underset{n \to \infty}{\longrightarrow} 0$, entonces $T_n$ es consistente para $\theta$
\end{definición}

\begin{proposición}
  Si $T_{n} = T(X_1, \ldots, X_n)$ es una sucesión de estimadores tales que $\forall \theta \in \Theta, E_{\theta}[T_n] \underset{n \to \infty}{\longrightarrow} \theta$, $V_{\theta}(T_n) \underset{n \to \infty}{\longrightarrow} 0$, entonces $T_n$ es consistente para $\theta$
\end{proposición}
\begin{proof}
  $E_{\theta}\left[\left(T_{n}-\theta\right)^{2}\right]=V_{\theta}\left(T_{n}\right)+b\left(T_{n}, \theta\right)^{2} \underset{n \rightarrow \infty}{\longrightarrow} 0, \forall \theta \in \Theta$ entonces $T_{n} \xrightarrow[n \rightarrow \infty]{\text { m.c. }} \theta, \forall \theta \in \Theta$ 
\end{proof}

\ejemplo{
  El estimador $a_k = \frac{1}{n}\sum_{i = 1}^{n}X_i^k$ es un estimdor consistente pra el mometo poblacional con respecto al origen de orden $k$, es decir, es estimador del parámetro $\theta = \alpha_k$. 
}

\ejemplo{
  Sea una m.a.s. de tamaño $n$ de una población de $Bernouilli(\theta)$ comprobemos que el estimador $T_n = \frac{1}{n+2}\left(\sum_{i = 1}^{n}X_i + 1\right)$ es un estimador consistente para $\theta$: 
  $$\lim_{n \to \infty} E[T_n] = \lim_{n \to \infty} E\left[\frac{1}{n+2}\sum_{i = 1}^{n}X_i + 1\right] = \lim_{n \to \infty} \frac{1}{n+2}\sum_{i = 1}^{n}(E[X_i] + 1) = \lim_{n \to \infty}\frac{n\theta + 1}{n+2} = \theta$$
  $$\lim_{n \to \infty} V[T_n] = \lim_{n \to \infty} V\left[\frac{1}{n+2}\sum_{i = 1}^{n}X_i + 1\right] = \lim_{n \to \infty} \frac{1}{(n+2)^2}\sum_{i = 1}^{n}V[X_i] = \lim_{n \to \infty} \frac{n\theta(1-\theta)}{(n+2)^2} = 0$$
}

\subsection{Estimadores Bayesianos}


\begin{definición}[Estimadores Bayesianos]
  El enfoque bayesiano trata a los parámetros de las distribuciones como variables aleatorias con su propia función de distribución, a diferencia de considerar que toma valores fijos desconocidos. Para desarrollar este punto de vista, se asigna una distribución a $\theta$ llamada \underline{distribución inicial o a priori} $\pi(\theta)$ y se actualiza esta distribución con la información de la muestra para obtener la \underline{distribución final o a posteriori} $\pi(\theta | x_1, \ldots, x_n)$
  $$\pi\left(\theta \mid x_{1}, \ldots, x_{n}\right)=\frac{\pi(\theta) f\left(x_{1}, \ldots, x_{n} \mid \theta\right)}{m\left(x_{1}, \ldots, x_{n}\right)}$$  
  donde m es la \underline{distribución predictiva}, dada por 
  $$m(x_1, \ldots, x_n) = \int_{\Theta} \pi(\theta)f(x_1, \ldots, x_n | \theta)d\theta$$
\end{definición}

\begin{observación}
  Antes de tomar la muestra, la información sobre $\theta$ viene dada por $\pi(\theta)$ y tras la experimentación se debe utilizar $\pi\left(\theta \mid x_{1}, \ldots, x_{n}\right)$. El estimador bayesiano de $\theta$ es toda la distribución final y por extensión cualquier medida de centralización correspondiente a esta distribución
\end{observación}

\ejemplo{
  Sea una m.a.s. de tamaño $n$ de una población $Bin(1, \theta)$ y con $\theta \sim U(0, 1)$ entonces $\pi(\theta | x_1, \ldots, x_n) \sim Beta\left(\sum_{i = 1}^{n}x_i + 1, n - \sum_{i = 1}^{n}x_i + 1\right)$
  $$ \theta \sim U(0, 1) \implies \pi(\theta) \frac{1}{1-0} = 1$$
  $$ X \sim Bin(1, \theta) \equiv Bernouilli(\theta) \implies f(x | \theta) = \theta^x(1-\theta)^{1-x} \implies $$ $$ \implies f(x_1, \ldots, x_n | \theta) = \prod_{i = 1}^{n}\theta^{x_i}(1-\theta)^{1-x_i} = \theta^{\sum_{i = 1}^{n}x_i}(1-\theta)^{n - \sum_{i = 1}^{n}x_i}$$
  $$ m(x_1, \ldots, x_n) = \int_{0}^{1} \pi(\theta)f(x_1, \ldots, x_n | \theta)d\theta = \int_{0}^{1} \theta^S(1-\theta)^{n-S}d\theta =  B(S + 1 n - S + 1) \implies$$
  $$ \pi(\theta | x_1, \ldots, x_n) = \frac{1 \cdot \theta^S (1-\theta)^{n-S}}{B(S + 1, n - S + 1)} \implies \pi(\theta | x_1, \ldots, x_n) \sim Beta(S + 1, n - S + 1)$$
  \\
  Ahora teniendo en cuenta que $U(0, 1) \equiv Beta(1, 1)$, podemos generalizar el resultado anterior para cualquier distribución inicial $\pi(\theta) \sim Beta(\alpha, \beta)$
  $$\theta \sim Beta(\alpha, \beta) \implies \pi(\theta) = \frac{\theta^{\alpha - 1}(1-\theta)^{\beta - 1}}{B(\alpha, \beta)}$$
  $$ m(x_1, \ldots, x_n) = \int_{0}^{1} \pi(\theta)f(x_1, \ldots, x_n | \theta)d\theta = \int_{0}^{1} \frac{\theta^{\alpha - 1}(1-\theta)^{\beta - 1}\theta^S(1-\theta)^{n-S}}{B(\alpha, \beta)}d\theta = $$
  $$ = \frac{B(\alpha + S, \beta + n - S)}{B(\alpha, \beta)} = \frac{\Gamma(\alpha + S)\Gamma(\beta + n - S)}{\Gamma(\alpha + \beta + n)} \implies \pi(\theta | x_1, \ldots, x_n) \sim Beta(\alpha + S, \beta + n - S)$$
  En este caso se dice que la familia de distribuciones iniciales $Beta(\alpha, \beta)$ es conjugada de la familia de distribuciones de probabilidad $X \sim Bin(1, \theta)$
  \begin{observación}
    Además, tenemos que $E[\theta | x_1, \ldots, x_n] = \frac{\sum_{i = 1}^{n}x_i + \alpha}{n + \alpha + \beta} = \frac{n}{n + \alpha + \beta}\bar{x} + \frac{\alpha + \beta}{n + \alpha + \beta}\frac{\alpha}{\alpha + \beta}$
  \end{observación}  
}

\ejemplo{
  Sea una m.a.s. de tamaño $n$ de una población $N(\mu, \sigma)$ y con $\mu \sim N(\mu_0, \sigma_0)$ y $\sigma$ conocida entonces $\pi(\mu | x_1, \ldots, x_n) \sim N(\mu_1, \sigma_1)$: 
  $$\mu \sim N(\mu_0, \sigma_0) \implies \pi(\mu) = \frac{1}{\sqrt{2\pi}\sigma_0}e^{-\frac{(\mu - \mu_0)^2}{2\sigma_0^2}}$$
  $$X \sim N(\mu, \sigma) \implies f(x | \mu) = \frac{1}{\sqrt{2\pi}\sigma}e^{-\frac{(x - \mu)^2}{2\sigma^2}} \implies f(x_1, \ldots, x_n | \mu) = \prod_{i = 1}^{n}\frac{1}{\sqrt{2\pi}\sigma}e^{-\frac{(x_i - \mu)^2}{2\sigma^2}} = $$
  $$ = \frac{1}{(2\pi\sigma^2)^{\frac{n}{2}}}e^{-\frac{\sum_{i = 1}^{n}(x_i - \mu)^2}{2\sigma^2}}$$
  $$\pi(\theta | x_1, \ldots, x_n) = \frac{\frac{1}{\sqrt{2\pi}\sigma_0}e^{-\frac{(\mu - \mu_0)^2}{2\sigma_0^2}} \cdot \frac{1}{(2\pi\sigma^2)^{\frac{n}{2}}}e^{-\frac{\sum_{i = 1}^{n}(x_i - \mu)^2}{2\sigma^2}}}{\int_{\mathbb{R}}\frac{1}{\sqrt{2\pi}\sigma_0}e^{-\frac{(\mu - \mu_0)^2}{2\sigma_0^2}} \cdot \frac{1}{(2\pi\sigma^2)^{\frac{n}{2}}}e^{-\frac{\sum_{i = 1}^{n}(x_i - \mu)^2}{2\sigma^2}}d\mu} = $$
  $$ = \frac{e^{-\frac{(\mu - \mu_0)^2}{2\sigma_0^2}} \cdot e^{-\frac{\sum_{i = 1}^{n}(x_i - \mu)^2}{2\sigma^2}}}{\int_{\mathbb{R}}e^{-\frac{(\mu - \mu_0)^2}{2\sigma_0^2}} \cdot e^{-\frac{\sum_{i = 1}^{n}(x_i - \mu)^2}{2\sigma^2}}d\mu} = \frac{e^{\frac{-\mu^2 - \mu_0^2 + 2\mu\mu_0}{2\sigma_0^2}} \cdot e^{\frac{-\sum_{i = 1}^{n}x_i^2 - n\mu^2 + 2\mu\sum_{i = 1}^{n}x_i}{2\sigma^2}}}{\int_{\mathbb{R}}e^{\frac{-\mu^2 - \mu_0^2 + 2\mu\mu_0}{2\sigma_0^2}} \cdot e^{\frac{-\sum_{i = 1}^{n}x_i^2 - n\mu^2 + 2\mu\sum_{i = 1}^{n}x_i}{2\sigma^2}}d\mu} = $$
  $$ = \frac{e^{\frac{-\mu^2 + 2\mu\mu_0}{2\sigma_0^2}} \cdot e^{\frac{-n\mu^2 + 2\mu\bar{X}n}{2\sigma^2}}}{\int_{\mathbb{R}} e^{\frac{-\mu^2 + 2\mu\mu_0}{2\sigma_0^2}} \cdot e^{\frac{-n\mu^2 + 2\mu\bar{X}n}{2\sigma^2}}d\mu} = \frac{e^{\frac{-\mu^2 + 2\mu\mu_0}{2\sigma_0^2}} \cdot e^{\frac{-n\mu^2 + 2\mu\bar{X}n}{2\sigma^2}}}{\int_{\mathbb{R}} e^{-\frac{1}{2}\left(\frac{1}{\sigma_0^2} + \frac{1}{\frac{\sigma^2}{n}}\right)\mu^2 + \left(\frac{\mu_0}{\sigma_0^2} + \frac{\bar{X}}{\frac{\sigma^2}{n}}\right)\mu}d} \implies$$
  Sabiendo que, dada una normal $N(E, \sqrt{V})$ se da que: $\int_{-\infty}^{+\infty}e^{-\frac{1}{2V}\theta^2 + \frac{E}{V}\theta}d\theta = \sqrt{2\pi V} \cdot e^{\frac{1}{2}\frac{E^2}{V}}$, entonces:
  $$\begin{cases}
  \frac{1}{V} = \frac{1}{\sigma_0^2} + \frac{1}{\frac{\sigma^2}{n}} \\
  \frac{E}{V} = \frac{\mu_0}{\sigma_0^2} + \frac{\bar{X}}{\frac{\sigma^2}{n}}
  \end{cases}$$
  $$ = \frac{e^{\frac{-\mu^2 + 2\mu\mu_0}{2\sigma_0^2}} \cdot e^{\frac{-n\mu^2 + 2\mu\bar{X}n}{2\sigma^2}}}{\sqrt{2\pi\left(\frac{1}{\sigma_0^2} + \frac{1}{\frac{\sigma^2}{n}}\right)} \cdot e^{\frac{1}{2}\left(\frac{\mu_0}{\sigma_0^2} + \frac{\bar{X}}{\frac{\sigma^2}{n}}\right)^2}} = \frac{e^{\frac{-\mu^2 + 2\mu\mu_0}{2\sigma_0^2}} \cdot e^{\frac{-n\mu^2 + 2\mu\bar{X}n}{2\sigma^2}}}{\sqrt{2\pi\left(\frac{1}{\sigma_0^2} + \frac{n}{\sigma^2}\right)} \cdot e^{\frac{1}{2}\left(\frac{\mu_0^2}{\sigma_0^2} + \frac{\bar{X}^2}{\frac{\sigma^2}{n}}\right)}} = $$

  NO ESTÁ TERMINADO
}

\begin{definición}[Estadístico suficiente bayesiano]
  $T = T(X_1, \ldots, X_n)$ es un estadístico suficiente bayesiano para $\theta$ para la familia $\mathcal{P} = \{f(\vec{x}|\theta) : \theta \in \Theta\}$ si cualquiera que sea la distribución inicial $\pi(\theta)$, se tiene que la distribución final dada por la muestra y por el valor del estadístico son la misma. Es decir: 
  $$\pi(\theta | x_1, \ldots, x_n) = \pi(\theta | t) : t = T(x_1, \ldots, x_n)$$
\end{definición}

\begin{teorema}[Versión bayesiana del Teorema de Factorización de Fisher]
    $T = T(X_1, \ldots, X_n)$ es un estadístico suficiente para $\theta$ si y sólo si $T$ es un estadístico suficiente bayesiano para $\theta$ respecto a $\pi(\theta)$, cualquiera que sea la distribución inicial $\pi(\theta)$
\end{teorema}

\begin{proof}
  $$\begin{aligned}
    &\Rightarrow \pi(\theta | x_1, \ldots, x_n) = \frac{\pi(\theta)f(x_1, \ldots, x_n | \theta)}{m(x_1, \ldots, x_n)} = \frac{\pi(\theta)g(t | \theta)f(x_1, \ldots, x_n | t, \theta)}{m(x_1, \ldots, x_n)} = \frac{\pi(\theta)g(t | \theta)}{\int_{\Theta}\pi(\theta)g(t | \theta)d\theta} = \pi(\theta | t)\\
    &\Rightarrow f(x_1, \ldots, x_n | \theta) = \frac{\pi(\theta | x_1, \ldots, x_n)m(x_1, \ldots, x_n)}{\pi(\theta)} = \frac{\pi(\theta | t)}{\pi(\theta)}m(x_1, \ldots, x_n)
  \end{aligned}$$
\end{proof}

\begin{definición}[Error cuadrático medio]
  Dado un estimador $T(X_1, \ldots, X_n)$ de $\theta$, se denomina error cuadrático medio $ECM$ a la expresión en función de $\theta$:	
  $$ECM_T(\theta) = E_{\theta}[(T - \theta)^2]$$
  Conceptualmente, el error cuadrático medio es una medida que indica qué tn cerca está un estadístico del parámetro verdadero que se intenta estimar. 
\end{definición}

\begin{proposición}
  Dado un estimador $T(X_1, \ldots, X_n)$ de $\theta$, se tiene que:
  $$ECM_T(\theta) = V_{\theta}(T) + b(T, \theta)^2$$
\end{proposición}
\begin{proof}
  $$ECM_T(\theta) = E_{\theta}[(T - \theta)^2] = E_{\theta}[T^2 - 2T\theta + \theta^2] = E_{\theta}[T^2] - 2\theta E_{\theta}[T] + \theta^2 = V_{\theta}(T) + b(T, \theta)^2$$
\end{proof}

\begin{observación}
  El sesgo mide qué tanto se desvía, en promedio, el estimador del valor verdadero del parámetro. \\
  La varianza del estimador mide cómo varían las estimaciones (del estimador) si tomamos diferentes muestras. \\
  Es decir, responden a las preguntas de ¿Apunta al lugar correcto? y ¿Qué tan dispersas están las estimaciones? respectivamente. 
\end{observación}

\ejemplo{
  Dada una m.a.s. de tamaño $n$ de una población $X \sim Bernouilli(\theta)$ el error cuadrático medio del estimador bayesiano $\bar{X}$: 
  $$E_{\theta}[(T - \theta)^2] = E_{\theta}[(\bar{X} - \theta)^2] = E_{\theta}[\bar{X}^2 - 2\bar{X}\theta + \theta^2] = E_{\theta}[\bar{X}^2] - 2\theta E_{\theta}[\bar{X}] + \theta^2 = $$ $$ = \frac{1}{n^2}\sum_{i = 1}^{n}E_{\theta}[X_i^2] - 2\theta \frac{1}{n} \sum_{i = 1}^{n}E_{\theta}[X_i] + \theta^2 = \frac{1}{n^2}n(\theta^2 + \theta(1-\theta)) - 2\theta^2 + \theta^2 = \frac{\theta(1-\theta)}{n} = \frac{\theta}{n} - \frac{\theta^2}{n} = \frac{\theta(1-\theta)}{n}$$
}

\ejemplo{
  Dada una m.a.s. de tamaño $n$ de una población $X \sim N(\mu, \sigma)$, se sab quelos estimadores centrados de ambos paráetros son $\bar{X}$ para $\mu$ y $S^2$ para $\sigma^2$, respectivamente. Y sus errores cuadráticos medios son: 
  $$ECM_{\bar{X}}(\mu) = Var[\bar{X}] = \frac{\sigma^2}{n} \quad ECM_{S^2}(\sigma^2) = Var[S^2] = \frac{2\sigma^4}{n-1}$$
  Sea $b_2 = \sigma_n^2 = \frac{1}{n} \sum_{i = 1}^{n}(X_i - \bar{X})^2$ otro estimador centrado para $\sigma^2$, calculemos su varianza y sesgo: 
  Gracias al cálculo realizado en un ejemplo anterior tenemos que $b(\sigma_n^2, \sigma) = -\frac{\sigma^2}{n}$, por lo que sólo queda calcular la varianza, la cual es de la forma
  $$V_{\theta}(\sigma_n^2) = \frac{2\sigma^4}{n} \implies ECM_{\sigma_n^2}(\sigma^2) = V_{\theta}(\sigma_n^2) + b(\sigma_n^2, \sigma)^2 = \frac{2\sigma^4}{n} - \frac{\sigma^4}{n^2} = \frac{2n - 1}{n^2}\sigma^4 < \frac{2\sigma^4}{n-1} = ECM_{S^2}(\sigma^2)$$
  Por lo que se puede concluir que $S^2$ es un estimador más eficiente que $\sigma_n^2$ para $\sigma^2$
  A pesar de que matemáticamente se parezcan mas, la corrección para mejorar la eficiencia se la conoce como \underline{corrección de Bessel}.
}

\begin{observación}
  En general, si $T_1$ y $T_2$ son dos estimadores de $\theta$ y $ECM_{T_1}(\theta) < ECM_{T_2}(\theta)$, entonces $T_1$ es un estimador más eficiente que $T_2$ para $\theta$
\end{observación}

\subsection{Criterios de comparación de estimadores}

\begin{definición}[Pérdida final esperada]
  Dado un estimador $T(X_1, \ldots, X_n)$, la distribución inicial $\pi(\theta)$ y la función de pérdida $\mathcal{L}(\theta, t)$ donde $t$ son los valores que toma el estimador, se define la Pérdida Final Esperada o $PFE$ o el riesgo a posteriori como: 
  $$PFE_T = E[\mathcal{L}(t, \theta) | X_1 = x_1, \ldots, X_n = x_n] = \int_{\Theta} \mathcal{L}(\theta, t)\pi(\theta | x_1, \ldots, x_n)d\theta$$
\end{definición}

\begin{proposición}
  Puede darse que existan varias funciones de pérdida, veamos las más comunes: 
  \begin{enumerate}
    \item Si $\mathcal{L}(\theta, t) = (\theta - t)^2$ entonces $PFE(t) = V(\theta | x_1, \ldots, x_n) + b(T, \theta)^2$ y la pérdida final esperada se minimimia en $t^* = E[\theta | x_1, \ldots, x_n]$
    \item Si $\mathcal{L}(\theta, t) = |\theta - t|$ entonces $PFE(t) = E[|\theta - t| | x_1, \ldots, x_n]$ y la pérdida final esperada se minimiza en la mediana de la distribución final (estimador bayesiano)
  \end{enumerate}
\end{proposición}

\begin{proof}
  \begin{enumerate}
    \item Si $\mathcal{L}(\theta, t) = (\theta - t)^2 \implies$ $$PFE_T = E[(\theta - t)^2 | x_1, \ldots, x_n] = E[((\theta - E[\theta | x_1, \ldots, x_n]) + (E[\theta | x_1, \ldots, x_n] - t))^2 | x_1, \ldots, x_n] $$ Si expandimos el cuadrado: 
    $$ (\theta - E[\theta | x_1, \ldots, x_n])^2 + 2(\theta - E[\theta | x_1, \ldots, x_n])(E[\theta | x_1, \ldots, x_n] - t) + (E[\theta | x_1, \ldots, x_n] - t)^2 \implies$$
    Calculemos cada una de las esperanzas por separado: 
    $$E[(\theta - E[\theta | x_1, \ldots, x_n])^2 | x_1, \ldots, x_n] = V(\theta | x_1, \ldots, x_n)$$
    $$E[(\theta - E[\theta | x_1, \ldots x_n] | x_1, \ldots, x_n)] = 0 \text{ por propiedades de la esperanza condicional}$$
    $$E[(E[\theta | x_1, \ldots, x_n] - t)^2 | x_1, \ldots, x_n] = (E[\theta | x_1, \ldots, x_n] - t)^2 \text{ dada una muestra, se vuelve una constante}$$
    Por lo que se puede concluir que $PFE_T = V(\theta | x_1, \ldots, x_n) + (E[\theta | x_1, \ldots, x_n] - t)^2$ y se minimiza en $t^* = E[\theta | x_1, \ldots, x_n]$
    \item Si $\mathcal{L}(\theta, t) = |\theta - t| \implies$ $$PFE_T = E[|\theta - t| | x_1, \ldots, x_n] = \int_{\Theta} |\theta - t|F(\theta | x)d\theta \implies$$ Dividiendo la integral entre los valores positivos y los negativos de $\theta$, nos queda que la integral es la suma de: 
    $$\int_{-\infty}^{t} (t - \theta)f(\theta | x)d\theta + \int_{t}^{+\infty} (\theta - t)f(\theta | x)d\theta$$
    Si derivamos con respecto a $t$ y obtenemos $0$ podemos ver un posible punto máximo o mínimo de la función: 
    $$\frac{d}{dt}PFE_T = \frac{d}{dt} \int_{-\infty}^{t} (t - \theta)f(\theta | x)d\theta + \frac{d}{dt} \int_{t}^{+\infty} (\theta - t)f(\theta | x)d\theta = $$ $$ = \int_{-\infty}^{t} f(\theta | x)d\theta - \int_{t}^{+\infty} f(\theta | x)d\theta = 0 \iff$$
    $$F_X(t) = 1 - F_X(t) \iff F_X(t) = \frac{1}{2} \implies t^* = \text{mediana de la distribución final}$$
  \end{enumerate}
\end{proof}

\ejemplo{
  EJEMPLO DE LA PÉRDIDA FINAL ESPERADA
}

\begin{definición}[Estimador centrado uniformemente de mínima varianza]
  $T^{*} = T^{*}(X_1, \ldots, X_n)$ es un estimador centrado uniformemente de mínima varianza para $\theta$ si y sólo si $E_{\theta}[T^{*}] = \theta$ y para cualquier otro estimador $T = T(X_1, \ldots, X_n)$ con $E_{\theta}[T] = \theta$, se tiene que $V_{\theta}(T^{*}) \leq V_{\theta}(T)$, $\forall \theta \in \Theta$
\end{definición}

\begin{proposición}
  Si existe un estimador centrado uniformemente de mínima varianza para $\theta$, entonces es único c.s.
\end{proposición}

\begin{proof}
  Sean $T_1$ y $T_2$ dos estimadores centrados uniformemente de mínima varianza para $\theta$, demostremos que entones $T_1 \overset{c.s.}{\equiv} T_2$. \\
  Sea $T = \frac{T_1 + T_2}{2} \implies E_{\theta}[T] = \theta$
  $$V_{\theta}(T) = V_{\theta}(\frac{T_1 + T_2}{2}) = \frac{1}{4}(V_{\theta}(T_1) + V_{\theta}(T_2) + 2Cov_{\theta}(T_1, T_2)) = \frac{V_{\theta}(T_1)}{2} + \frac{Cov(T_1, T_2)}{2}$$
  Sabemos que la correlación de dos variables aleatorias está acotada por 1, entonces: 
  $$\rho_{\theta}(T_1, T_2) = \frac{Cov(T_1, T_2)}{\sqrt{V_{\theta}(T_1)V_{\theta}(T_2)}} \leq 1 \iff Cov(T_1, T_2) \leq V_{\theta}(T_1) \implies$$
  $$\implies V_{\theta}(T) \leq V_{\theta}(T_1) $$
  Pero además como $T_1$ es un estimador centrado uniformemente de mínima varianza, ningún otro estimador centrado puede tener un avarianza más pequeña: $V_{\theta}(T) \leq V_{\theta}(T_1)$
  Por lo tanto $V_{\theta}(T) = V_{\theta}(T_1) = Cov_{\theta}(T_1, T_2) = V_{\theta}(T_2) \implies \rho_{\theta}(T_1, T_2) = 1 \implies \exists a, b : T_1 = aT_2 + b \iff E[T_1] = aE[T_2] + b \iff \theta = a\theta + b \iff a = 1, b = 0 \implies T_1 \overset{c.s.}{\equiv} T_2$
\end{proof}

\ejemplo{
  Sea una m.a.s. de tamaño $n$ de una población $X \sim N(\mu, \sigma^2)$ vamos a trabajar con la familia de estimadores $T_k = \{k \cdot S_n^2\}$. Calculemos cuál es el menor error cuadrático medio. Y tomemos como función del estimador $d(\theta) = \theta^2$
  $$ECM_{T_k}(\theta) = V_{\theta}(T_k) + b(T_k, \theta)^2 \implies$$ $$ \begin{cases} b(T_k, \theta) = E_{\theta}[T_k] - d(\theta)  = \frac{k\theta^2}{n} \cdot E_{\theta}[\frac{n}{\theta^2}S_n^2] =  \frac{k\theta^2}{n} \cdot E_{\theta}[\chi_{n}] - \theta^2 = \frac{k\theta^2}{n}\cdot n - \theta^2  = k\theta^2 - \theta = (k-1)\theta^2\\ V_{\theta}(T_k) = V_{\theta}[k \cdot S_n^2] = \frac{k^2\theta^4}{n^2}V_{\theta}[\frac{n}{\theta^2}S_n^2] = \frac{k^2\theta^4}{n^2}V_{\theta}[\chi_n^2] = \theta^4 \cdot \frac{k^2}{n^2} \cdot 2n = \frac{2\theta^4k^2}{n}\end{cases} \implies $$ $$ \implies ECM_{T_k}(\theta) = \frac{2\theta^4k^2}{n} + (k-1)^2\theta^4 = \theta^4 \left(\frac{2k^2}{n} + (k-1)^2\right)$$
  Para encontrar el valor de $k$ que minimiza el error cuadrático medio, derivamos con respecto a $k$ e igualamos a 0:
  $$\frac{d}{dk}ECM_{T_k}(\theta) = 0 \iff \frac{d}{dk}\left(\theta^4 \left(\frac{2k^2}{n} + (k-1)^2\right)\right) = 0 \iff \theta^4 \left(\frac{4k}{n} + 2(k-1)\right) = 0 \iff$$
  $$\iff \frac{4k}{n} + 2(k-1) = 0 \iff 4k + 2n(k-1) = 0 \iff 4k + 2nk - 2n = 0 \iff 4k + 2nk = 2n \iff$$  $$\iff k(4 + 2n) = 2n \iff k = \frac{2n}{4 + 2n} = \frac{n}{2 + n}$$
  Por lo que el estimador que minimiza el error cuadrático medio es $T_{\frac{n}{2 + n}} = \frac{n}{2 + n}S_n^2$
}

\begin{teorema}
  El estimador centrado uniformemente de mínima varianza es función simétrica de las observaciones
\end{teorema}

\ejemplo{
  Sea una m.a.s. de tamaño $n = 2$, entonces el estimador $T_1 = \frac{X_1}{X_2}$ no puede ser un estimador centrado uniformemente de mínima varianza, ya que si lo fuera, entonces para el nuevo estimador $T_2 = \frac{X_2}{X_1}$ se tendría que $E_{\theta}[T_2] = E_{\theta}[T_1]$ y $V_{\theta}(T_2) = V_{\theta}(T_1)$ con lo que el estimador $T = \frac{T_1 + T_2}{2} = \frac{X_1^2 + X_2^2}{X_1X_2}$ sería tal que $V_{\theta} < V_{\theta}(T_1)$, lo cual es una contradicción
}

\begin{observación}
  En general si tienes un estimador $T$ que no es simétrico, puedes promediar sobre todas las permutaciones posibles para crear un nuevo estimador $\bar{T}$: 
  $$\bar{T} = \frac{1}{n!}\sum_{i = 1}^{n!} T_j$$
  Este nuevo estimador es simétrico respecto a las observaciones y $V_{\theta}(\bar{T}) \leq V_{\theta}(T_j) \forall j$ donde se cumple que $V_{\theta}(\bar{T}) < V_{\theta}(T_j)$ si $T_j$ no es un estimador simétrico. Además, se cumple que $E_{\theta}[\bar{T}] = E_{\theta}[T]$	
\end{observación}

\begin{teorema}[Teorema de caracterización del ECUMV]
  Sea $T_1 = T_1(X_1, \ldots, X_n)$ un estimador centrado para $\theta$ ($E_{\theta}[T_1] = \theta$) y $V_{\theta}(T_1) < \infty$ entonces $T_1$ es el ECUMV para $\theta$ si y sólo si para cualquier otro estimador $T_2 = T_2(X_1, \ldots, X_n)$ con $E_{\theta}[T_2] = 0$ y $V_{\theta}(T_2) < \infty$ se tiene que $E_{\theta}[T_1T_2] = 0$
\end{teorema}

\begin{corolario}
  Si $T_1 = T_2(X_1, \ldots X_n)$ y $T_2 = T_2(X_1, \ldots, X_n)$ son ECUMV para $h_1(\theta)$ y $h_2(\theta)$ respectivamente, entonces $b_1T_1 + b_2T_2$ es el ECUMV para $b_1h_1(\theta) + b_2h_2(\theta)$
\end{corolario}

\begin{teorema}[Teorema de Rao-Blackwell]
  Si $T = T(X_1, \ldots, X_n)$ es un estimador centrado para $\theta$ con $V_{\theta}(T) < \infty$ y $S(X_1, \ldots, X_n)$ es un estadístico suficiente, entonces $g(S) = E[T | S]$ es un estimador centrado para $\theta$ con $V_{\theta}(g(S)) \leq V_{\theta}(T)$
\end{teorema}

\begin{proof}
  Si $S$ es una estadística suficiente para el parámetro $\theta$, entonces $E[T \mid S]$ no depende de $\theta$. Por las propiedades de la esperanza condicionada, se tiene que:

  \[
  E_{\theta}[g(S)] = E_{\theta}[E[T \mid S]] = E_{\theta}[T] = \theta
  \]

  Ahora, considerando la varianza de $T$:

  \[
  V_{\theta}(T) = E_{\theta}[(T - \theta)^2] = E_{\theta}[(T - g(S) + g(S) - \theta)^2]
  \]

  Expandiendo el cuadrado y usando la linealidad de la esperanza:

  \[
  V_{\theta}(T) = E_{\theta}[(T - g(S))^2] + E_{\theta}[(g(S) - \theta)^2] + 2E_{\theta}[(T - g(S))(g(S) - \theta)]
  \]

  El último término se anula debido a la siguiente propiedad de la esperanza condicionada:

  \[
  E_{\theta}[(T - g(S))(g(S) - \theta)] = \iint (t - g(s))(g(s) - \theta) dF_{\theta}(t, s)
  \]

  Descomponiendo en términos de la distribución condicional:

  \[
  = \int (g(s) - \theta) \left( \int (t - g(s)) dF(t \mid s) \right) dF_{\theta}(s) = 0
  \]

  Ya que $E[T \mid S] = g(S)$, la esperanza condicional centrada es cero.

  Por lo tanto,

  \[
  V_{\theta}(T) = V_{\theta}(g(S)) + E_{\theta}[(T - g(S))^2] \geq V_{\theta}(g(S))
  \]

  donde se alcanza la igualdad si y solo si $T = g(S)$ casi seguramente.
\end{proof}


\begin{teorema}[Teorema de Lehmann-Schefeé]
  Si $S(X_1, \ldots, S_n)$ es un estadístico suficiente y completo para $\theta$ y $T = T(X_1, \ldots, X_n)$ es un estimador centrado para $\theta$ tal que $T = h(S)$, entonces $T$ es ECUMV para $\theta$
\end{teorema}

\begin{proof}
  $\begin{cases} S \text{ suficiente } \\ T \text{ centrado } \end{cases} \implies g(S) = E[T|S] \text{ es centrado para } \theta \text{ y } V_{\theta}(g(S)) \leq V_{\theta}(T)$ \\
  Además, se tiene que para cualquier otro estimador $T_1$ centrado para $\theta$, $g_1(S) = E[T_1 | S]$ es centrado para $\theta$ y $V_{\theta}(g_1(S)) \leq V_{\theta}(T_1)$ \\
  Por lo tanto al ser $T$ completo y $E_{\theta}[g(S) - g_1(S)] = \theta - \theta = 0$ se tiene que $g(S) \stackrel{c.s.}{=} g_1(S)$. En particular, para $T_1 = T = h(S)$, $g_1(S) = E[h(S) | S] = h(S) = T$ y $V_{\theta}(T) \leq V_{\theta}(T_1)$, cualquiera que sea $T_1$ centrado para $\theta$
\end{proof}

\ejemplo{
  Sean  una m.a.s. de tamaño $n$ de una población con $X \sim Bin(1, \theta)$ y un estadístico $T = \sum_{i = 1}^{n} X_i$, comprueba que es suficiente y completo y además, si $h(T) = \bar{X}$, comprueba entonces que $h(T)$ es el ECUMV para $\theta$: \\
  Veamos primero que $T = \sum_{i = 1}^{n} X_i$ es suficiente y completo para $\theta$:	
  $$X \sim Bin(1, \theta) \equiv Bernouilli(\theta) \implies f_{\theta}(x) = \theta^x(1 - \theta)^{1-x} \implies f_{\theta}(x_1, \ldots, x_n) = \theta^{\sum_{i = 1}^{n} x_i}(1 - \theta)^{n - \sum_{i = 1}^{n} x_i}$$
  Entones por el Teorema de Factorización de Fisher, tenemos que $T$ es suficiente para $\theta$. Veamos ahora su completitud: \\
  $$ X \sim Bin(1, \theta) \implies T \sim Bin(n, \theta) \implies$$
  Sea $g$ función real tal que: $E_{\theta}[g(T)] = 0, \forall \theta \in [0, 1]$, entonces:
  $$E_{\theta}[g(T)] = \sum_{t = 0}^{n} g(t) \binom{n}{t} \theta^t(1 - \theta)^{n-t} = 0, \forall \theta \in [0, 1] \iff$$
  $$\iff (1 - \theta)^n \cdot \sum_{i = 1}^{n} g(t) \binom{n}{t} \left(\frac{\theta}{1 - \theta}\right)^t = 0 \iff$$
  $$\iff \sum_{t = 0}^{n} g(t) \binom{n}{t} x^t = 0, \forall x \in \mathbb{R} \iff g(t) = 0, \forall t \in \{0, \ldots, n\} $$  ya que los coeficientes binomiales son no nulos. \\
  Por último queda ver que $S$ es un estadístico centrado para $\theta$, i.e. $E_{\theta}[h(S)] = \theta$ y $V_{\theta}(h(S)) < \infty$:
  $$E_{\theta}[h(T)] = E_{\theta}[\bar{X}] = \frac{1}{n} \cdot \sum_{i = 1}^{n} E_{\theta}[X_i] = \frac{1}{n} \cdot n\theta = \theta$$
  $$V_{\theta}(h(T)) = V_{\theta}(\bar{X}) = \frac{1}{n^2} \cdot \sum_{i = 1}^{n} V_{\theta}(X_i) = \frac{1}{n^2} \cdot n\theta(1 - \theta) = \frac{\theta(1 - \theta)}{n} < \infty$$
  Por lo tanto, $T = n\bar{X}$ es el ECUMV para $\theta$
}

\ejemplo{
  Sea una m.a.s. de tamaño $n$ de una población con distribución $Poisson(\theta)$ y dado un estadistico $d(\theta) = e^{-2\theta}$. Encuentra el ECUMV para $d(\theta)$:\\ 
}

\ejemplo{
  Sea una m.a.s. de tamaño $n$ de una población que sigue una distribución $Poisson(\theta)$, tenemos que encontrar el ECUMV para $\theta$: \\
  Si tomamos el estadístico $S = \sum_{i = 1}^{n} X_i$, para poder aplicar el Teorema de Lehmann-Scheffé necesitamos ver que el estadístico sea completo y suficiente: \\

  Veamos primero que $S$ es suficiente para $\theta$: \\
  $$
  X \sim Poisson(\theta) \implies f_{\theta}(x) = \frac{e^{-\theta}\theta^x}{x!} \implies f_{\theta}(x_1, \ldots, x_n) = \frac{e^{-n\theta}\theta^{\sum_{i = 1}^{n} x_i}}{\prod_{i = 1}^{n} x_i!}
  $$

  Entonces, por el Teorema de Factorización de Fisher, tenemos que $S$ es suficiente para $\theta$. Veamos ahora su completitud: \\

  Siguiendo con lo anterior:
  $$
  f_{\theta}(x_1, \ldots, x_n) = \frac{e^{-n\theta}\theta^{\sum_{i = 1}^{n} x_i}}{\prod_{i = 1}^{n} x_i!} = \frac{e^{-n\theta} e^{\ln(\theta) \sum x_i}}{\prod_{i = 1}^{n} x_i!} \implies
  \begin{cases}
    c(\theta)^n = e^{-n\theta} \\
    \prod_{i = 1}^{n} h(x_i) = \prod_{i = 1}^{n} x_i! \\
    q_1(\theta) = \ln(\theta) \\
    T_1(\vec{x}) = \sum_{i = 1}^{n} x_i
  \end{cases}
  $$
  Entonces, debemos ver que $ln(\theta)$ contiene un abierto  $(0, +\infty) \subset \mathbb{R}$, por lo tanto $s$ es completo para $\theta$.
  Además, tomando el estadístico $T = \frac{1}{n} \sum_{i = 1}^{n}x_i$ tenemos que: 
  $$E[T] = \frac{1}{n} \sum_{i = 1}^{n} E[X_i] = \frac{1}{n} \cdot n\theta = \theta \implies \text{ T es centrado para } \theta \implies $$
  Tomando la función $h(x) = \frac{1}{n}\cdot x$ tenemos que: $$h(S) = \frac{1}{n} \sum_{i = 1}^{n} X_i = \frac{S}{n} \implies$$
  Por el Teorema de Lehmann-Scheffé, $T = h(S)$ es el ECUMV para $\theta$.
  \\
  Si en lugar de haber tomado $d(\theta) = \theta$ hubieramos querido el estimador centrado uniformemente de mínima varianza para $d(\theta) = e^{-\theta}$ HAY QUE INSERTAR LO DE DIEGO
}

\ejemplo{
  Sea una m.a.s. de tamaño $n$ de una población que sigue una distribución exponencial con parámetro $\theta$, queremos encontrar el estimador centrado uniformemente de mínima varianza para $d(\theta) = \theta$. \\
  $$X \sim Exponencial(\theta) \implies f(x|\theta) = \theta e^{-\theta x} \implies f(x_1, \ldots, x_n | \theta) = \theta^n e^{-\theta \sum_{i = 1}^{n} x_i}$$
  Por el Teorema de Factorización de Fisher, $S = \sum_{i = 1}^{n} X_i$ es suficiente para $\theta$. Veamos ahora su completitud: \\
  Como se sigue una distribución exponencial, podemos ver que pertenece a una familia exponencial uniparamétrica: 
  $$f(x_1, \ldots, x_n | \theta) = \theta^n e^{-\theta \sum_{i = 1}^{n} x_i} \implies \begin{cases} c(\theta)^n = \theta^n \\ h(\vec{x}) = 1 \\ q_1(\theta) = -\theta \\ T_1(\vec{x}) = \sum_{i = 1}^{n} x_i \end{cases}$$
  Por lo que evidentemente en la imagen de la función $q_1(\theta) = -\theta$ tiene un abierto en su imagen por lo que el estadístico $S$ es completo para $\theta$. \\
  Por último, sabemos que $\bar{X} = \frac{S}{n} \implies E[\bar{X}] = \frac{1}{\theta} \implies \bar{X}$ es centrado para $\frac{1}{\theta}$ y por el Teorema de Lehmann-Scheffé, $\bar{X}$ es el ECUMV para $\frac{1}{\theta}$.\\
  Pero nosotros lo que queríamos es un estimador centrado para $\theta$. Por lo que puede parecer intuitivo pensar que el estadístico que podría estar centrado para $\theta$ es $\frac{1}{\bar{X}}$: 
  $$E[\frac{1}{\bar{X}}] = n \cdot E[\frac{1}{S}] = n \cdot E[\frac{1}{\sum_{i = 1}^{n}X_i}] = n \cdot \frac{n-1}{\theta} \implies S' = \frac{n(n-1)}{\sum_{i = 1}^{n}X_i}$$
  Por lo que $S'$ es el ECUMV para $\theta$
}

\ejemplo{
  Sea una m.a.s. de tamaño $n$ de una población que sigue una distribución $Bernouilli(\theta)$, busquemos cual es el estimador centrado uniformemente de mínima varianza para $d_1(\theta) = \theta$ y para $d_2(\theta) = \theta(1 - \theta)$: \\
  $$X \sim Bernouilli(\theta) \implies f(x|\theta) = \theta^x(1 - \theta)^{1-x} \implies f(x_1, \ldots, x_n | \theta) = \theta^{\sum_{i = 1}^{n} x_i}(1 - \theta)^{n - \sum_{i = 1}^{n} x_i}$$
  Por el Teorema de Factorización de Fisher, $S = \sum_{i = 1}^{n} X_i$ es suficiente para $\theta$. Veamos ahora su completitud: \\
  $$f(x_1, \ldots, x_n | \theta) = \theta^{\sum_{i = 1}^{n} x_i}(1 - \theta)^{n - \sum_{i = 1}^{n} x_i} = (1 - \theta)^n \cdot e^{\sum x_i \cdot ln(\frac{\theta}{1 - \theta})} \implies \begin{cases}
    c(\theta)^n = (1 - \theta)^n \\
    h(\vec{x}) = 1 \\
    q_1(\theta) = ln(\frac{\theta}{1 - \theta}) \\
    T_1(\vec{x}) = \sum_{i = 1}^{n} x_i
  \end{cases}$$
  Por lo que evidentemente en la imagen de la función $q_1(\theta) = ln(\frac{\theta}{1 - \theta})$ tiene un abierto en su imagen por lo que el estadístico $S = \sum_{i = 1}^{n} X_i$ es completo para $\theta$.
  \begin{enumerate}
    \item Primero veamos el caso para $d_1(\theta) = \theta$: Sabemos que $E[S] = \sum_{i = 1}^{n}E[X_i] = n\theta \implies \bar{X} = \frac{S}{n} \implies E[\bar{X}] = \theta \implies \bar{X}$ es centrado para $\theta$. Por el Teorema de Lehmann-Scheffé, $\bar{X}$ es el ECUMV para $\theta$
    \item Ahora veamos el caso para $d_2(\theta) = \theta(1 - \theta) = \theta - \theta^2$: A pesar de que pueda parecer lógico tomar la varianza muestral $S^2$ como estimador, no se cumplen las condiciones del Teorema de Lehmann-Scheffé para que sea un ECUMV, ya que tiene que depender únicamente de $S = \sum X_i$ y en $S^2$ aparecen términos cuadráticos, por lo que no es un ECUMV. Podemos intentar buscar otro:\\
    NO LO ENTIENDO
  \end{enumerate}
}

\ejemplo{
  Sea una m.a.s. de tamao $n$ de una población que sigue una distribución $Normal(\mu, \sigma^2)$, veamos distintas casuísticas: 
  \begin{enumerate}
    \item Tomemos uque $\mu$ es conocida y $\sigma^2$ no, por lo que queremos buscar el ECUMV para $\sigma^2$:
    $$X \sim Normal(\mu, \sigma^2) \implies f(x|\sigma^2) = \frac{1}{\sqrt{2\pi\sigma^2}}e^{-\frac{(x - \mu)^2}{2\sigma^2}} \implies f(x_1, \ldots, x_n | \sigma^2) = \frac{1}{(2\pi\sigma^2)^{\frac{n}{2}}}e^{-\frac{\sum_{i = 1}^{n}(x_i - \mu)^2}{2\sigma^2}}$$
    Por el Teorema de Factorización de Fisher, $S = \sum_{i = 1}^{n} (X_i - \mu)^2$ es suficiente para $\sigma^2$. Veamos ahora su completitud: \\
    Sabemos que la distribución normal, en este caso, pertenece a la familia exponencial uniparamétrica. Entonces sabemos que los estadísticos naturales asociados a las funciónes paramétricas $q_i(\theta)$ son completos, si éstas ultimas contienen un abierto en su imagen. En este caso, $q_1(\sigma^2) = \frac{1}{2\sigma^2}$ contiene un abierto en su imagen, por lo que $S$ es completo para $\sigma^2$. \\
    Veamos la esperanza de $S$: 
    $$E[S] = E[\sum_{i = 1}^{n} (X_i - \mu)^2] = \sum_{i = 1}^{n} E[(X_i^2 + \mu^2 - 2X_i\mu)] = \sum_{i = 1}^{n} E[X_i^2] + \sum_{i = 1}^{n} \mu^2 - \sum_{i = 1}^{n} 2\mu E[X_i] = $$ $$ = n(\sigma^2 + \mu^2) + n\mu^2 - 2n\mu^2 = n\sigma^2 \implies \frac{S}{n} \text{ es el ECUMV para } \sigma^2 \text{ por el T. de Lehmann-Schefeé}$$
    \item Ahora supongamos que $\mu$ y $\sigma^2$ son desconocidos, por lo que queremos encontrar el ECUMV para $\mu$ y $\sigma^2$:
    $$f(x|\mu, \sigma^2) = \frac{1}{\sqrt{2\pi\sigma^2}}e^{-\frac{(x - \mu)^2}{2\sigma^2}} \implies f(x_1, \ldots, x_n | \mu, \sigma^2) = \frac{1}{(2\pi\sigma^2)^{\frac{n}{2}}}e^{-\frac{\sum_{i = 1}^{n}(x_i - \mu)^2}{2\sigma^2}} = $$
    $$ = f(x_1, \ldots, x_n | \mu, \sigma^2) = \frac{1}{(2\pi\sigma^2)^{\frac{n}{2}}}e^{-\frac{\sum x_i^2 + n\mu^2 - \mu\sum x_i}{2\sigma^2}}$$
    Por el Teorema de Factorización de Fisher, $T(\vec{x}) = (\sum x_i, \sum x_i^2)$ es suficiente para $(\mu, \sigma^2)$. Veamos ahora su completitud: \\
    Sabemos que la distribución normal, en este caso, pertenece a la familia exponencial uniparamétrica. Entonces sabemos que los estadísticos naturales asociados a las funciónes paramétricas $q_i(\theta)$ son completos, si éstas ultimas contienen un abierto en su imagen. En este caso, $q_1(\mu, \sigma^2) = \mu$ y $q_2(\mu, \sigma^2) = \frac{1}{2\sigma^2}$ contienen un abierto en su imagen, por lo que $T$ es completo para $(\mu, \sigma^2)$. \\
    Veamos la esperanza de $T$:
    $$E[T] = E[(\sum X_i, \sum X_i^2)] = (\sum E[X_i], \sum E[X_i^2]) = (n\mu, n\sigma^2 + n\mu^2)$$ $$\implies \frac{\sum X_i}{n} = \bar{X} \text{ es el ECUMV para } \mu \text{ y } S^2 = \frac{1}{n-1}\sum_{i = 1}^{n}(X_i - \bar{X})^2 \text{ es el ECUMV para } \sigma^2$$
    \item Ahora suponamos que $\mu$ es desconocido y $\sigma^2$ es conocido, por lo que queremos encontrar el ECUMV para $\mu$:
    Por lo visto en el apartado anterior, podemos que por el Teorema de Fisher, podemos ver que el estadítico $T = \sum_{i = 1}^{n} X_i$ es suficiente para $\mu$. \\También podríamos preguntarnos porqué no sería el estadístico $\vec{T} = (\sum X_i, \sum X_i^2)$, la respuesta está en que esta distribución normal pertenece a la familia exponencial uniparamétrica y por tanto tomamos la función dependiente del parámetro $q_1(\mu) = \mu$ y su estadístico natural asociado $T_1(\vec{x}) = \sum X_i$, el cual por las propiedades que vimos de las familias exponenciales es suficiente, y además completo. 
    Veamos ahora si es insesgado: 
    $$E[T] = E[\sum X_i] = \sum E[X_i] = n\mu \implies \bar{X} = \frac{T}{n} \text{ es el ECUMV para } \mu$$
   \end{enumerate}
}

\subsection{Cota para la varianza de un estimador}
Hemos visto hasta ahora el \underline{estimador centrado uniformemente de mínima varianza}, que cómo su nombre indica, es el que tiene varianza más pequeña.  \\
Si definieramos una varianza mínima para los estimadores de un parámetro, es decir, si encontrásemos una cota para los estimadores, podríamos encontrar más fácilmente un estimador que tenga dicha varianza y por tanto seria el centrado uniformemente de mínima varianza. 
\begin{definición}[Condiciones de regularidad de Wolfowitz]
  Sea $X \approx\left(\chi, \beta_{\chi}, F_{\theta}\right)_{\theta \in \Theta \subset \mathbb{R}}$ modelo estadístico uniparamétrico contínuo (o discreto) y sea ( $X_{1}, \cdots X_{n}$ ) muestra de $\left\{F_{\theta}, \theta \in \Theta\right\}$ siendo $f_{\theta}\left(x_{1}, \cdots, x_{n}\right)$ su función de densidad (o de masa). Supongamos que se verifican las siguientes condiciones de regularidad:
  \begin{enumerate}
    \item $\Theta$ es un intervalo abierto de $\mathbb{R}$
    \item $\operatorname{Sop}\left(f_{\theta}\right)=\left\{\left(x_{1}, \cdots, x_{n}\right) \in \chi^{n}: f_{\theta}\left(x_{1}, \cdots, x_{n}\right)>0\right\}$ no depende de $\theta$
    \item $\forall\left(x_{1}, \cdots, x_{n}\right) \in \chi^{n}$ y $\forall \theta \in \Theta, \exists \frac{\partial}{\partial \theta} f_{\theta}\left(x_{1}, \cdots, x_{n}\right)$
    \item $\int_{\chi^{n}} \frac{\partial}{\partial \theta} f_{\theta}\left(x_{1}, \cdots, x_{n}\right) d x_{1} \cdots d x_{n}=0$
    \item $I_{n}(\theta)=E\left[\left(\frac{\partial}{\partial \theta} \log f_{\theta}\left(x_{1}, \cdots, x_{n}\right)\right)^{2}\right]<\infty($ cantidad de información de Fisher)
  \end{enumerate}
\end{definición}

\begin{teorema}[Cota de Fréchet-Cramér-Rao]
  
\end{teorema}


 (continuación)Teorema (Cota de Fréchet-Cramér-Rao) Si $T=T\left(X_{1}, \cdots, X_{n}\right)$ es un estadístico unidimensional tal que $E_{\theta}\left[T^{2}\right]<\infty, E_{\theta}[T]=d(\theta)$ y $d^{\prime}(\theta)=\int_{\chi^{n}} T\left(x_{1}, \cdots, x_{n}\right) \frac{\partial}{\partial \theta} f_{\theta}\left(x_{1}, \cdots, x_{n}\right) d x_{1} \cdots d x_{n}$, entonces $d^{\prime}(\theta)^{2} \leq V_{\theta}(T) I_{n}(\theta), \forall \theta \in \Theta$, con igualdad si y solo si existe una función $k(\theta)$ tal que

$$
  P_{\theta}\left(\left(x_{1}, \cdots, x_{n}\right) \in x^{n}: T\left(x_{1}, \cdots, x_{n}\right)=d(\theta)+k(\theta) \frac{\partial}{\partial \theta} f_{\theta}\left(x_{1}, \cdots, x_{n}\right)\right)=1, \forall \theta \in \theta
$$

Demostración $\exists d^{\prime}(\theta)$ puesto que

$$
  \begin{gathered}
    d^{\prime}(\theta)=\int_{\chi^{n}} T\left(x_{1}, \cdots, x_{n}\right) \frac{\partial}{\partial \theta} \log \left(f_{\theta}\left(x_{1}, \cdots, x_{n}\right)\right) f_{\theta}\left(x_{1}, \cdots, x_{n}\right) d x_{1} \cdots d x_{n}=E_{\theta}\left[T \frac{\partial}{\partial \theta} \log f_{\theta}\right] \\
    \left|d^{\prime}(\theta)\right| \leq E_{\theta}\left[\left|T \frac{\partial}{\partial \theta} \log f_{\theta}\right|\right] \leq \sqrt{E_{\theta}\left[T^{2}\right] E_{\theta}\left[\left(\frac{\partial}{\partial \theta} \log f_{\theta}\right)^{2}\right]}<\infty \text { (desigualdad de Cauchy-Swartz) }
  \end{gathered}
$$

\section*{Cota para la varianza de un estimador (continuación)}
Además, $E_{\theta}\left[\frac{\partial}{\partial \theta} \log f_{\theta}\left(x_{1}, \cdots, x_{n}\right)\right]=0$ y por lo tanto,\\
$V_{\theta}\left[\frac{\partial}{\partial \theta} \log f_{\theta}\left(x_{1}, \cdots, x_{n}\right)\right]=E_{\theta}\left[\left(\frac{\partial}{\partial \theta} \log f_{\theta}\left(x_{1}, \cdots, x_{n}\right)\right)^{2}\right]=I_{n}(\theta)$\\
En efecto, $0=\int_{x^{n}} \frac{\partial}{\partial \theta} f_{\theta}\left(x_{1}, \cdots, x_{n}\right) d x_{1} \cdots d x_{n}=\int_{x^{n}} \frac{\partial}{\partial \theta} \log \left(f_{\theta}\left(x_{1}, \cdots, x_{n}\right)\right) f_{\theta}\left(x_{1}, \cdots, x_{n}\right) d x_{1} \cdots d x_{n}$

$$
  =E_{\theta}\left[\frac{\partial}{\partial \theta} \log f_{\theta}\left(x_{1}, \cdots, x_{n}\right)\right]
$$

Entonces, $\operatorname{Cov}_{\theta}\left[T, \frac{\partial}{\partial \theta} \log f_{\theta}\right]=E\left[T \frac{\partial}{\partial \theta} \log f_{\theta}\right]=d^{\prime}(\theta)$, y como $\rho_{\theta}^{2}\left(T, \frac{\partial}{\partial \theta} \log f_{\theta}\right)=\frac{d^{\prime}(\theta)^{2}}{V_{\theta}(T) V_{\theta}\left(\frac{\partial}{\partial \theta} \log f_{\theta}\right)} \leq 1$, se tiene que\\
$d^{\prime}(\theta)^{2} \leq V_{\theta}(T) I_{n}(\theta)$, con igualdad si y sólo si $\rho_{\theta}^{2}\left(T, \frac{\partial}{\partial \theta} \log f_{\theta}\right)=1$, es decir, si y sólo si $T \stackrel{\text { c.s. }}{=} a+b \frac{\partial}{\partial \theta} \log f_{\theta}$, es decir, si y sólo si existe una función $k(\theta)$ tal que $P_{\theta}\left(T=d(\theta)+k(\theta) \frac{\partial}{\partial \theta} f_{\theta}\right)=1$

\section*{Cota para la varianza de un estimador (continuación)}
En efecto, si $T \stackrel{\text { c.s. }}{=} a+b \frac{\partial}{\partial \theta} \log f_{\theta}$, entonces $d(\theta)=E_{\theta}[T]=a y$

$$
  d^{\prime}(\theta)=E_{\theta}\left[T \frac{\partial}{\partial \theta} \log f_{\theta}\right]=E_{\theta}\left[a \frac{\partial}{\partial \theta} \log f_{\theta}+b\left(\frac{\partial}{\partial \theta} \log f_{\theta}\right)^{2}\right]=b I_{n}(\theta),
$$

y $b=\frac{d^{\prime}(\theta)}{l_{n}(\theta)}=k(\theta)$\\
Proposición 2 Bajo las suposiciones anteriores, si T es un estadístico tal que $E_{\theta}[T]=d(\theta)$ y $V_{\theta}(T)=\frac{d^{\prime}(\theta)^{2}}{l_{n}(\theta)}$, entonces $T$ es ECUMV para d $(\theta)$

Proposición 3 Bajo las suposiciones anteriores, si ( $X_{1}, \cdots X_{n}$ ) es m.a.s. (n) de $\left\{F_{\theta}, \theta \in \Theta\right\}$, entonces $I_{n}(\theta)=n I_{1}(\theta)$ Indicación: $f_{\theta}\left(x_{1}, \cdots x_{n}\right)=\prod_{i=1}^{n} f_{\theta}\left(x_{i}\right)$

\section*{Cota para la varianza de un estimador (continuación)}
Proposición 4 Bajo las suposiciones anteriores, si la distribución de $X$ pertenece a la familia exponencial uniparamétrica, es decir, $f_{\theta}(x)=c(\theta) h(x) e^{q_{1}(\theta) T_{1}(x)}$, con $q_{1}^{\prime}(\theta)$ no nula, entonces el estadístico $\frac{1}{n} \sum_{i=1}^{n} T_{1}\left(X_{i}\right)$ alcanza la cota de FCR para $d(\theta)=-\frac{c^{\prime}(\theta)}{c(\theta) q_{1}^{\prime}(\theta)}$

\section*{Demostración}
$f_{\theta}\left(x_{1}, \ldots, x_{n}\right)=c(\theta)^{n} \prod_{i=1}^{n} h\left(x_{i}\right) e^{q_{1}(\theta) \sum_{i=1}^{n} T_{1}\left(x_{i}\right)}$\\
$\frac{\partial}{\partial \theta} \log f_{\theta}=n \frac{c^{\prime}(\theta)}{c(\theta)}+q_{1}^{\prime}(\theta) \sum_{i=1}^{n} T_{1}\left(x_{i}\right)$\\
$\frac{1}{n} \sum_{i=1}^{n} T_{1}\left(x_{i}\right)=a(\theta)+b(\theta) \frac{\partial}{\partial \theta} \log f_{\theta}, a(\theta)=-\frac{c^{\prime}(\theta)}{c(\theta) q_{1}^{\prime}(\theta)}, b(\theta)=\frac{1}{n q_{1}^{\prime}(\theta)}$\\
$\frac{1}{n} \sum_{i=1}^{n} T_{1}\left(x_{i}\right)$ es centrado para $d(\theta)=-\frac{c^{\prime}(\theta)}{c(\theta) q_{1}^{\prime}(\theta)}$ y alcanza la cota

\section*{Cota para la varianza de un estimador (continuación)}
Ejercicio Si $X \sim \operatorname{Bin}(1, \theta), T=\bar{X}$ alcanza la cota de FCR para $d(\theta)=\theta$

Ejercicio Si se cumplen las condiciones de regularidad y además\\
(1) $\forall\left(x_{1}, \cdots, x_{n}\right) \in \chi^{n} y \forall \theta \in \Theta, \exists \frac{\partial^{2}}{\partial \theta^{2}} f_{\theta}\left(x_{1}, \cdots, x_{n}\right)$\\
(2) $\int_{\chi^{n}} \frac{\partial^{2}}{\partial \theta^{2}} f_{\theta}\left(x_{1}, \cdots, x_{n}\right) d x_{1} \cdots d x_{n}=0$

Entonces, $I_{n}(\theta)=-E\left[\frac{\partial^{2}}{\partial \theta^{2}} \log f_{\theta}\left(x_{1}, \cdots, x_{n}\right)\right]$\\
Indicación: $\frac{\partial}{\partial \theta} \log f_{\theta}\left(x_{1}, \cdots, x_{n}\right)=\frac{\partial}{\partial \theta} f_{\theta}\left(x_{1}, \cdots, x_{n}\right) \frac{1}{f_{\theta}\left(x_{1}, \cdots, x_{n}\right)}$

\section*{Cota para la varianza de un estimador}
 (continuación)\section*{Estimadores eficientes}
Diremos que un estimador es eficiente para $d(\theta)$ si es centrado para $d(\theta)$ y su varianza alcanza la cota de FCR\\
En general, se llama eficiencia de un estimador centrado de $d(\theta)$ a

$$
  e f(T, d(\theta))=\frac{d^{\prime}(\theta)^{2}}{I_{n}(\theta) V_{\theta}(T)} \leq 1
$$

\section*{Métodos de construcción de estimadores}
Método de los momentos\\
Este método consiste en elegir como estimador de un momento poblacional su momento muestral asociado, es decir\\
(1) El estimador por el método de los momentos del momento poblacional respecto al origen de orden $\mathrm{k}, \alpha_{k}=E\left[X^{k}\right]$, es $a_{k}=\frac{1}{n} \sum_{i=1}^{n} X_{i}^{k}$\\
(2) El estimador por el método de los momentos del momento poblacional respecto a la media de orden k ,

$$
  \beta_{k}=E\left[\left(X-\alpha_{1}\right)^{k}\right], \text { es } b_{k}=\frac{1}{n} \sum_{i=1}^{n}\left(X_{i}-\bar{X}\right)^{k}
$$

Ejercicio Si $X \sim \operatorname{Gamma}(a, p)$, calcular un estimador por el método de los momentos de $\theta=(a, p)$ basado en una m.a.s. ( $n$ )

\section*{Métodos de construcción de estimadores}
\section*{(continuación)}
\section*{Método de máxima verosimilitud}
Supongamos que una urna contiene 6 bolas entre blancas y negras, no todas del mismo color, pero se ignora cuantas hay de cada uno. Para tratar de adivinar la composición de la urna se permiten dos extracciones con reemplazamiento de la misma y resultó que ninguna de ellas fue blanca. Dar una estimación de la probabilidad $\theta$ de que una bola extraída aleatoriamente de dicha urna sea blanca

$$
  \theta \in \Theta=\left\{\frac{1}{6}, \frac{2}{6}, \frac{3}{6}, \frac{4}{6}, \frac{5}{6}\right\}
$$

$T=X_{1}+X_{2} \equiv \mathrm{n}^{\circ}$ de blancas en las dos extracciones C.R. de la urna $\sim \operatorname{Bin}(2, \theta)$ y $f_{\theta}(t)=\binom{2}{t} \theta^{t}(1-\theta)^{2-t}, t=0,1,2$

\section*{Métodos de construcción de estimadores}
 (continuación)\begin{center}
  \begin{tabular}{|l|l|l|l|l|l|}
    \hline
    $\theta$                       & $1 / 6$ & $2 / 6$ & $3 / 6$ & $4 / 6$ & $5 / 6$ \\
    \hline
    $f_{\theta}(0)=(1-\theta)^{2}$ & 0.694   & 0.444   & 0.25    & 0.111   & 0.027   \\
    \hline
  \end{tabular}
\end{center}

Por lo tanto, la estimación $\hat{\theta}(0)=\frac{1}{6}$ y el estimador

$$
  \hat{\theta}=\hat{\theta}(T)=\left\{\begin{array}{lll}
    1 / 6 & \text { si } & T=0 \\
    1 / 2 & \text { si } & T=1 \\
    5 / 6 & \text { si } & T=2
  \end{array}\right.
$$

es el estimador de máxima verosimilitud (EMV)

\section*{Métodos de construcción de estimadores}
 (continuación)Sea $\left(X_{1}, \cdots X_{n}\right)$ una muestra con $f_{\theta}\left(x_{1}, \cdots, x_{n}\right)=f\left(x_{1}, \cdots, x_{n} \mid \theta\right)$ función de densidad (o de masa), $\theta \in \Theta \subset \mathbb{R}^{\ell}$ Denotemos por $L\left(\theta \mid x_{1}, \cdots, x_{n}\right)=f\left(x_{1}, \cdots, x_{n} \mid \theta\right)$ a la función de verosimilitud de la muestra Un estimador $\hat{\theta}=\hat{\theta}\left(X_{1}, \cdots, X_{n}\right)$ se denomina estimador de máxima verosimilitud (EMV) de $\theta$, sí y sólo sí\\
(1) $\hat{\theta}\left(x_{1}, \cdots, x_{n}\right) \in \Theta, \forall\left(x_{1}, \cdots, x_{n}\right) \in \chi^{n}$\\
(2) $L\left(\hat{\theta} \mid x_{1}, \cdots, x_{n}\right)=\sup _{\theta \in \Theta} L\left(\theta \mid x_{1}, \cdots, x_{n}\right), \forall\left(x_{1}, \cdots, x_{n}\right) \in \chi^{n}$\\
o equivalentemente, sí y sólo sí\\
(1) $\hat{\theta}\left(x_{1}, \cdots, x_{n}\right) \in \Theta, \forall\left(x_{1}, \cdots, x_{n}\right) \in \chi^{n}$\\
(2) $\log L\left(\hat{\theta} \mid x_{1}, \cdots, x_{n}\right)=\sup _{\theta \in \Theta} \log L\left(\theta \mid x_{1}, \cdots, x_{n}\right), \forall\left(x_{1}, \cdots, x_{n}\right) \in \chi^{n}$

\section*{Métodos de construcción de estimadores (continuación)}
Si $f_{\theta}$ es una función derivable respecto a $\theta$ en el interior del espacio paramétrico $\Theta$, la forma usual de determinar el estimador de máxima verosimilitud es examinar primero los máximos relativos de $f_{\theta}$, para compararlos después, con los valores sobre la frontera de $\Theta$. Ello conduce a resolver las ecuaciones de verosimilitud:

$$
  \frac{\partial}{\partial \theta_{j}} \log L\left(\theta \mid x_{1}, \cdots, x_{n}\right)=0, j=1, \cdots, \ell
$$

(en el supuesto de que $\theta=\left(\theta_{1}, \cdots, \theta_{\ell}\right)$ sea un parámetro $\ell$-dimensional), seleccionando las soluciones correspondientes a un máximo de $f_{\theta}$, es decir aquellas en las que la matriz hessiana $H=\left(\frac{\partial}{\partial \theta_{i}} \frac{\partial}{\partial \theta_{j}} \log L\left(\theta \mid x_{1}, \cdots, x_{n}\right)\right)_{i, j=1, \cdots, \ell}$ sea definida negativa

\section*{Métodos de construcción de estimadores}
 (continuación)

\section*{Observaciones}
 (1) EI EMV $\hat{\theta}$ no tiene por qué existir\\
(2) El EMV $\hat{\theta}$ no tiene por qué ser único\\
(3) El EMV $\hat{\theta}$ no tiene por qué ser centrado\\
(4) El EMV $\hat{\theta}$ no tiene por qué ser suficiente, pero si $S=S\left(X_{1}, \cdots, X_{n}\right)$ es suficiente para $\theta$, entonces $\hat{\theta}=\hat{\theta}(S)$\\
(5) Invariancia: Si $\hat{\theta}$ es el EMV de $\theta$, entonces $h(\hat{\theta})$ es el EMV de $h(\theta)$\\
(6) Bajo ciertas condiciones de regularidad, si $\left(X_{1}, \cdots X_{n}\right)$ es m.a.s. $(\mathrm{n})$ y $\theta \in \mathbb{R}$, entonces $\sqrt{n}(\hat{\theta}-\theta) \underset{n \rightarrow \infty}{d} N\left(0, \frac{1}{\sqrt{l_{1}(\theta)}}\right)$ y por lo tanto, $\hat{\theta}$ es asintóticamente insesgado para $\theta$ y asintóticamente eficiente
